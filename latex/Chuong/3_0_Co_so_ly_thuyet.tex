\documentclass[../DoAn.tex]{subfiles}
\begin{document}
Chương này tôi sẽ trình bày nền tảng lý thuyết và thuật toán được sử dụng để xây dựng ứng dụng gợi ý phim, đó là phương pháp Content-Based Filtering. Đây là phương pháp phổ biến và hiệu quả trong đề xuất nội dung cho người dùng dựa trên dữ liệu hiện có. Nội dung sẽ đi sâu vào các kiến thức nền tảng, cơ sở lý thuyết, các thuật toán cụ thể, và phương pháp nghiên cứu của phương pháp gợi ý. Mỗi phần sẽ phân tích rõ cách thức áp dụng thuật toán này để giải quyết các yêu cầu và vấn đề đã được xác định trong Chương 2. Qua đó, sẽ có cái nhìn tổng quan và chi tiết về cách thức hoạt động cũng như lợi ích của phương pháp gợi ý trong ứng dụng.
\section{Content-Based Filtering}
\label{section:3.1}
\subsection{Khái niệm}
Content-Based Filtering là một kỹ thuật gợi ý trong hệ thống đề xuất, dựa trên các đặc trưng của các đối tượng cần đề xuất. Thay vì dựa vào sở thích và hành vi của người dùng khác, CBF sử dụng các thuộc tính và đặc điểm cụ thể của sản phẩm để đưa ra gợi ý. Điều này giúp hệ thống có thể gợi ý các sản phẩm tương tự dựa trên những sản phẩm mà người dùng đã quan tâm hoặc đánh giá cao. Ví dụ, nếu một người dùng thích một bộ phim có thể loại hành động và có sự tham gia của một diễn viên cụ thể, hệ thống sẽ gợi ý các bộ phim khác có chủ đề tương tự và có sự tham gia của diễn viên đó.

\subsection{Các thuật toán}
Các thuật toán chính được sử dụng trong Content-Based Filtering bao gồm:

\textbf{TF-IDF (Term Frequency-Inverse Document Frequency):} Thuật toán này được sử dụng để đánh giá tầm quan trọng của từ khóa đó trong văn bản. Tần suất xuất hiện của từ trong văn bản (Term Frequency) và mức độ phổ biến của từ trong toàn bộ tập văn bản (Inverse Document Frequency) được kết hợp để xác định trọng số của từ khóa đó.

\textbf{Cosine Similarity:} Đây là một phương pháp đo độ tương tự giữa hai vector trong không gian vector. Độ tương tự cosine này được tính toán bằng cách tính giá trị cosin của góc giữa hai vector. Phương pháp này thường được sử dụng để đo độ tương tự giữa hồ sơ của người dùng và hồ sơ của các sản phẩm.

Việc áp dụng Content-Based Filtering không chỉ giúp ứng dụng gợi ý phim hoạt động hiệu quả mà còn đảm bảo rằng người dùng nhận được các đề xuất phù hợp với sở thích và nhu cầu cá nhân.

Chương này đã cung cấp một cái nhìn tổng quan và chi tiết về phương pháp gợi ý phim: Content-Based Filtering. Qua các phần trình bày về kiến thức nền tảng, cơ sở lý thuyết, các thuật toán và phương pháp nghiên cứu, chúng ta đã hiểu rõ cách thức hoạt động, ưu điểm và hạn chế của phương pháp.

Content-Based Filtering dựa vào phân tích các đặc trưng của nội dung phim để gợi ý các bộ phim tương tự mà người dùng đó có thể quan tâm. Phương pháp này giúp ứng dụng đưa ra các gợi ý chính xác dựa trên thông tin cụ thể của phim và sở thích cá nhân của người dùng.

Việc hiểu và áp dụng đúng các kỹ thuật của phương pháp này sẽ giúp ứng dụng gợi ý phim hoạt động hiệu quả, mang lại trải nghiệm tốt hơn cho người dùng. Các kiến thức và phương pháp trình bày trong chương này sẽ là nền tảng vững chắc để phát triển và tối ưu hóa ứng dụng gợi ý phim trong các chương tiếp theo.
\end{document}