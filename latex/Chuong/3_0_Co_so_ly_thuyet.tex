\documentclass[../DoAn.tex]{subfiles}
\begin{document}
Chương này được dành để giới thiệu về các nền tảng lý thuyết cũng như các thuật toán chính được sử dụng trong việc phát triển hệ thống gợi ý phim, với trọng tâm là phương pháp Content-Based Filtering. Đây được xem là một trong các kỹ thuật được sử dụng rộng rãi và đạt được hiệu quả đáng kể trong lĩnh vực đề xuất nội dung cá nhân hóa cho người dùng, dựa trên nguồn dữ liệu hiện có. Chương sẽ đi sâu vào việc giải thích các khái niệm nền tảng, các nguyên lý lý thuyết, các thuật toán cụ thể được triển khai, cùng với các phương pháp nghiên cứu có liên quan đến kỹ thuật gợi ý này. Mỗi nội dung sẽ giải thích rõ ràng về cách thức ứng dụng các thuật toán nhằm thỏa mãn các yêu cầu và khắc phục các vấn đề đã được nêu ra ở Chương 2. Nhờ vậy, người đọc có thể hiểu rõ về cơ chế hoạt động cũng như các lợi ích mà phương pháp gợi ý này đem lại cho hệ thống.
\section{Content-Based Filtering}
\label{section:3.1}
\subsection{Khái niệm}
Content-Based Filtering là một phương pháp gợi ý được triển khai trong các hệ thống đề xuất, hoạt động bằng cách khai thác và phân tích các đặc điểm cũng như các thuộc tính của các đối tượng cần được đề xuất. Điểm khác biệt so với các phương pháp gợi ý dựa trên sở thích và hành vi của cộng đồng người dùng, CBF chủ yếu dựa vào các thuộc tính bên trong và các đặc trưng độc đáo của mỗi sản phẩm để hình thành các gợi ý có tính phù hợp cao. Với phương pháp tiếp cận này, hệ thống có thể xác định và đề xuất các sản phẩm mang tính tương đồng cao dựa trên các sản phẩm mà người dùng đã bày tỏ sự quan tâm hoặc đưa ra đánh giá tốt. Ví dụ, nếu như một người dùng yêu thích một bộ phim thuộc thể loại hành động và có sự tham gia của một diễn viên cụ thể, khi đó hệ thống sẽ tự động tìm kiếm và đề xuất các bộ phim khác có cùng thể loại và có sự tham gia của diễn viên đó.

\subsection{Các thuật toán}
Các thuật toán chính được sử dụng trong Content-Based Filtering bao gồm:

\textbf{TF-IDF (Term Frequency-Inverse Document Frequency):} Thuật toán này được áp dụng nhằm xác định mức độ quan trọng của các từ khóa trong một văn bản. Việc tính toán dựa trên sự kết hợp giữa tần suất xuất hiện của từ trong văn bản (Term Frequency) và nghịch đảo của tần suất xuất hiện của từ đó trong toàn bộ tập văn bản (Inverse Document Frequency), từ đó xác định được trọng số phù hợp cho mỗi từ khóa.

\textbf{Cosine Similarity:} Đây là một phương pháp được sử dụng để đo lường mức độ tương đồng giữa hai vector trong không gian vector. Việc tính toán độ tương tự cosine được thực hiện thông qua việc xác định giá trị cosin của góc được tạo bởi hai vector đó. Kỹ thuật này thường được vận dụng để đánh giá mức độ tương đồng giữa hồ sơ người dùng và hồ sơ của các sản phẩm trong hệ thống.

Việc áp dụng Content-Based Filtering vào hệ thống giúp tối ưu hóa hiệu suất làm việc cũng như đảm bảo rằng mỗi người dùng đều nhận được các đề xuất phim phù hợp với sở thích và nhu cầu riêng của từng cá nhân.

Chương này đã trình bày một cách toàn diện và chi tiết về phương pháp gợi ý phim Content-Based Filtering. Thông qua việc trình bày các kiến thức cơ bản, nền tảng lý thuyết, các thuật toán cụ thể và các phương pháp nghiên cứu liên quan, chúng ta đã có được sự hiểu biết sâu sắc về nguyên lý hoạt động, những ưu điểm cũng như những hạn chế của phương pháp này.

Content-Based Filtering hoạt động dựa trên cơ sở phân tích các đặc trưng và thuộc tính của nội dung phim để từ đó đề xuất các bộ phim có tính chất tương đồng mà người dùng có khả năng quan tâm. Phương pháp này cho phép ứng dụng tạo ra các gợi ý có độ chính xác cao dựa trên các thông tin cụ thể về phim và các sở thích mang tính cá nhân của từng người dùng.

Việc hiểu rõ và áp dụng một cách chính xác các kỹ thuật của phương pháp này sẽ tạo điều kiện để cải thiện đáng kể hiệu suất vận hành của hệ thống gợi ý phim, qua đó tạo ra những trải nghiệm người dùng tốt hơn. Những kiến thức và phương pháp được trình bày trong chương này là cơ sở nền tảng để tôi có thể tiếp tục phát triển hệ thống gợi ý phim trong các chương tiếp theo.
\end{document}