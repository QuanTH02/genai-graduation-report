\documentclass[../DoAn.tex]{subfiles}
\begin{document}
Chương này tập trung trình bày các nền tảng lý thuyết và các thuật toán được áp dụng trong quá trình xây dựng ứng dụng gợi ý phim, cụ thể là phương pháp Content-Based Filtering. Phương pháp này được đánh giá là một trong những kỹ thuật phổ biến và mang lại hiệu quả cao trong việc đề xuất nội dung phù hợp với người dùng dựa trên các dữ liệu sẵn có. Nội dung chương sẽ trình bày chi tiết về các kiến thức cơ bản, nền tảng lý thuyết, các thuật toán cụ thể được sử dụng, và các phương pháp nghiên cứu liên quan đến kỹ thuật gợi ý này. Mỗi phần sẽ làm rõ cách thức vận dụng các thuật toán để đáp ứng các yêu cầu và giải quyết các vấn đề đã được đề cập trong Chương 2. Thông qua đó, người đọc sẽ có được cái nhìn toàn diện và sâu sắc về nguyên lý hoạt động cũng như những ưu điểm mà phương pháp gợi ý này mang lại cho ứng dụng.
\section{Content-Based Filtering}
\label{section:3.1}
\subsection{Khái niệm}
Content-Based Filtering là một kỹ thuật gợi ý được áp dụng trong các hệ thống đề xuất, hoạt động dựa trên việc phân tích các đặc trưng và thuộc tính của các đối tượng cần được đề xuất. Khác với các phương pháp dựa trên sở thích và hành vi của cộng đồng người dùng khác, CBF tập trung vào việc sử dụng các thuộc tính nội tại và các đặc điểm riêng biệt của từng sản phẩm để tạo ra các gợi ý phù hợp. Nhờ cách tiếp cận này, hệ thống có khả năng đề xuất các sản phẩm có tính chất tương đồng dựa trên những sản phẩm mà người dùng đã thể hiện sự quan tâm hoặc đánh giá tích cực. Chẳng hạn, khi một người dùng thể hiện sự yêu thích đối với một bộ phim thuộc thể loại hành động và có sự góp mặt của một diễn viên nhất định, hệ thống sẽ tự động đề xuất các bộ phim khác có cùng chủ đề và có sự xuất hiện của diễn viên đó.

\subsection{Các thuật toán}
Các thuật toán chính được sử dụng trong Content-Based Filtering bao gồm:

\textbf{TF-IDF (Term Frequency-Inverse Document Frequency):} Thuật toán này được áp dụng nhằm xác định mức độ quan trọng của các từ khóa trong một văn bản. Việc tính toán dựa trên sự kết hợp giữa tần suất xuất hiện của từ trong văn bản (Term Frequency) và nghịch đảo của tần suất xuất hiện của từ đó trong toàn bộ tập văn bản (Inverse Document Frequency), từ đó xác định được trọng số phù hợp cho mỗi từ khóa.

\textbf{Cosine Similarity:} Đây là một phương pháp được sử dụng để đo lường mức độ tương đồng giữa hai vector trong không gian vector. Việc tính toán độ tương tự cosine được thực hiện thông qua việc xác định giá trị cosin của góc được tạo bởi hai vector đó. Kỹ thuật này thường được vận dụng để đánh giá mức độ tương đồng giữa hồ sơ người dùng và hồ sơ của các sản phẩm trong hệ thống.

Việc vận dụng Content-Based Filtering không những góp phần nâng cao hiệu quả hoạt động của ứng dụng gợi ý phim mà còn đảm bảo người dùng nhận được những đề xuất phù hợp với sở thích riêng và các nhu cầu cá nhân của họ.

Chương này đã trình bày một cách toàn diện và chi tiết về phương pháp gợi ý phim Content-Based Filtering. Thông qua việc trình bày các kiến thức cơ bản, nền tảng lý thuyết, các thuật toán cụ thể và các phương pháp nghiên cứu liên quan, chúng ta đã có được sự hiểu biết sâu sắc về nguyên lý hoạt động, những ưu điểm cũng như những hạn chế của phương pháp này.

Content-Based Filtering hoạt động dựa trên cơ sở phân tích các đặc trưng và thuộc tính của nội dung phim để từ đó đề xuất các bộ phim có tính chất tương đồng mà người dùng có khả năng quan tâm. Phương pháp này cho phép ứng dụng tạo ra các gợi ý có độ chính xác cao dựa trên các thông tin cụ thể về phim và các sở thích mang tính cá nhân của từng người dùng.

Việc nắm vững và vận dụng đúng đắn các kỹ thuật của phương pháp này sẽ góp phần nâng cao hiệu quả hoạt động của ứng dụng gợi ý phim, từ đó mang lại những trải nghiệm tích cực hơn cho người dùng. Các kiến thức và phương pháp được trình bày trong chương này sẽ đóng vai trò là nền tảng quan trọng để tiếp tục phát triển và tối ưu hóa ứng dụng gợi ý phim trong các phần tiếp theo của báo cáo.
\end{document}