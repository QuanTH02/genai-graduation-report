\documentclass[../DoAn.tex]{subfiles}
\begin{document}
Chương này trình bày nền tảng lý thuyết và các thuật toán được sử dụng để xây dựng hệ thống gợi ý phim, bao gồm hai phương pháp chính là Content-Based Filtering và Collaborative Filtering. Đây là các phương pháp phổ biến và hiệu quả trong việc đề xuất nội dung cho người dùng dựa trên dữ liệu hiện có. Nội dung chương sẽ đi sâu vào các kiến thức nền tảng, cơ sở lý thuyết, các thuật toán cụ thể, và phương pháp nghiên cứu của từng phương pháp gợi ý. Mỗi phần sẽ phân tích rõ cách thức áp dụng từng thuật toán để giải quyết các yêu cầu và vấn đề đã được xác định trong Chương 2. Qua đó, sẽ có cái nhìn tổng quan và chi tiết về cách thức hoạt động cũng như lợi ích của từng phương pháp gợi ý trong hệ thống.
\section{Content-Based Filtering}
\label{section:3.1}
\subsection{Kiến thức nền tảng}
Content-Based Filtering (CBF) là một kỹ thuật gợi ý trong hệ thống đề xuất, dựa trên các đặc trưng của các đối tượng cần đề xuất. Thay vì dựa vào hành vi và sở thích của người dùng khác, CBF sử dụng các thuộc tính và đặc điểm cụ thể của sản phẩm để đưa ra gợi ý. Điều này giúp hệ thống có thể gợi ý các sản phẩm tương tự dựa trên những sản phẩm mà người dùng đã quan tâm hoặc đánh giá cao. Ví dụ, nếu một người dùng thích một bộ phim có chủ đề khoa học viễn tưởng và có sự tham gia của một diễn viên cụ thể, hệ thống sẽ gợi ý các bộ phim khác có chủ đề tương tự và có sự tham gia của diễn viên đó.

\subsection{Cơ sở lý thuyết}
Cơ sở lý thuyết của Content-Based Filtering dựa trên việc phân tích và mô tả các đặc trưng của sản phẩm. Mỗi sản phẩm được mô tả bằng một tập hợp các thuộc tính đặc trưng, ví dụ như thể loại, diễn viên, đạo diễn, hoặc từ khóa mô tả nội dung phim. Các đặc trưng này được sử dụng để tạo ra một hồ sơ (profile) cho mỗi sản phẩm. Hồ sơ của người dùng được xây dựng dựa trên các đặc trưng của những sản phẩm mà người dùng đã thích hoặc đánh giá cao. Hệ thống sử dụng các hồ sơ này để tìm kiếm và gợi ý các sản phẩm tương tự. Một số kỹ thuật thường được sử dụng trong CBF bao gồm vector hóa từ khóa (TF-IDF), phân tích tương quan Cosine, và các phương pháp học máy như K-Nearest Neighbors (KNN) để tính toán độ tương tự giữa các hồ sơ.

\subsection{Các thuật toán}
Các thuật toán chính được sử dụng trong Content-Based Filtering bao gồm:

\textbf{TF-IDF (Term Frequency-Inverse Document Frequency):} Thuật toán này được sử dụng để đánh giá tầm quan trọng của từ khóa trong văn bản. Tần suất xuất hiện của từ trong văn bản (Term Frequency) và mức độ phổ biến của từ trong toàn bộ tập văn bản (Inverse Document Frequency) được kết hợp để xác định trọng số của từ khóa.

\textbf{Cosine Similarity:} Đây là một phương pháp đo độ tương tự giữa hai vector trong không gian vector. Độ tương tự cosine này được xác định bằng cách tính giá trị cosin của góc giữa hai vector. Phương pháp này thường được sử dụng để đo độ tương tự giữa hồ sơ của người dùng và hồ sơ của sản phẩm.

\textbf{K-Nearest Neighbors (KNN):} Đây là một thuật toán học máy dùng để phân loại và hồi quy. Trong CBF, KNN được sử dụng để tìm ra k sản phẩm tương tự nhất với sản phẩm mà người dùng đã đánh giá cao.

\textbf{Naive Bayes:} Đây là một thuật toán phân loại dựa trên lý thuyết xác suất Bayes. Naive Bayes có thể được sử dụng để dự đoán xác suất một sản phẩm thuộc một loại cụ thể dựa trên các đặc trưng của nó.

\subsection{Phương pháp nghiên cứu}
Để triển khai Content-Based Filtering, cần thực hiện các bước sau:

\textbf{Thu thập dữ liệu:} Dữ liệu về các đặc trưng của sản phẩm cần được thu thập từ các nguồn uy tín như IMDb và BoxOfficeMojo. Dữ liệu này bao gồm các thuộc tính của phim như thể loại, diễn viên, đạo diễn, và các từ khóa mô tả nội dung.

\textbf{Xây dựng hồ sơ sản phẩm và người dùng:} Sử dụng các đặc trưng đã thu thập để xây dựng hồ sơ chi tiết cho mỗi sản phẩm. Đồng thời, xây dựng hồ sơ người dùng dựa trên các sản phẩm mà họ đã đánh giá cao hoặc yêu thích.

\textbf{Tính toán độ tương tự:} Sử dụng các thuật toán như TF-IDF và Cosine Similarity để tính toán độ tương tự giữa hồ sơ của người dùng và hồ sơ của các sản phẩm.

\textbf{Gợi ý sản phẩm:} Dựa trên độ tương tự đã tính toán, gợi ý các sản phẩm tương tự mà người dùng có thể quan tâm.

\textbf{Đánh giá và cải thiện:} Đánh giá hiệu quả của hệ thống gợi ý thông qua các chỉ số như độ chính xác, độ bao phủ, và độ tin cậy. Dựa trên kết quả đánh giá, điều chỉnh các tham số của thuật toán và cập nhật dữ liệu để cải thiện hệ thống.

Việc áp dụng Content-Based Filtering không chỉ giúp hệ thống gợi ý phim hoạt động hiệu quả mà còn đảm bảo rằng người dùng nhận được các đề xuất phù hợp với sở thích và nhu cầu cá nhân.

\section{Collaborative Filtering}
\label{section:3.2}
\subsection{Kiến thức nền tảng}
Collaborative Filtering (CF) là một kỹ thuật gợi ý dựa trên sự hợp tác giữa các người dùng trong hệ thống. Phương pháp này không chỉ dựa trên đặc trưng của các sản phẩm mà còn dựa trên hành vi và sở thích của người dùng. Cụ thể, CF sử dụng thông tin từ các người dùng khác có sở thích tương tự để đưa ra gợi ý. Đây là một trong những phương pháp gợi ý phổ biến nhất và được áp dụng rộng rãi trong các hệ thống khuyến nghị như Amazon, Netflix, và Spotify.

\subsection{Cơ sở lý thuyết}
Phương pháp Collaborative Filtering (CF) là một trong những phương pháp quan trọng trong lọc cộng tác trong lĩnh vực học máy và khai phá dữ liệu. Cơ sở lý thuyết của CF dựa trên hai giả định chính là User similarity và Item similarity. CF được triển khai thông qua hai phương pháp chính:

User-Based Collaborative Filtering tìm kiếm những người dùng có sở thích tương tự với người dùng hiện tại và gợi ý những sản phẩm mà họ đã thích nhưng người dùng hiện tại chưa khám phá.

Ngược lại, Item-Based Collaborative Filtering tìm kiếm các sản phẩm tương tự dựa trên các sản phẩm mà người dùng đã đánh giá cao và gợi ý các sản phẩm tương tự này cho người dùng.

\subsection{Các thuật toán}
Các thuật toán chính được sử dụng trong Collaborative Filtering bao gồm:

\textbf{User-Based Collaborative Filtering (UBCF):} Thuật toán này tính toán độ tương tự giữa các người dùng bằng cách so sánh lịch sử đánh giá của họ. Một phương pháp phổ biến để đo độ tương tự là sử dụng Cosine Similarity hoặc Pearson Correlation. Sau khi tìm ra những người dùng có độ tương tự cao, hệ thống sẽ gợi ý các sản phẩm mà họ đã thích nhưng người dùng hiện tại chưa đánh giá.

\textbf{Item-Based Collaborative Filtering (IBCF):} Thay vì so sánh giữa các người dùng, IBCF so sánh giữa các sản phẩm. Hệ thống sẽ tìm ra các sản phẩm tương tự dựa trên lịch sử đánh giá của tất cả người dùng. Các phương pháp như Cosine Similarity hoặc Jaccard Similarity có thể được sử dụng để đo độ tương tự giữa các sản phẩm.

\textbf{Matrix Factorization:} Đây là một kỹ thuật học máy nâng cao được sử dụng rộng rãi trong CF. Matrix Factorization, như SVD (Singular Value Decomposition), phân rã ma trận đánh giá người dùng-sản phẩm thành các ma trận nhỏ hơn, từ đó phát hiện ra các yếu tố ẩn định nghĩa sự tương đồng giữa người dùng và sản phẩm.

\textbf{k-Nearest Neighbors (k-NN):} Thuật toán k-NN có thể được áp dụng cả trong UBCF và IBCF. Trong UBCF, k-NN tìm ra k người dùng tương tự nhất, trong khi ở IBCF, nó tìm ra k sản phẩm tương tự nhất.

\subsection{Phương pháp nghiên cứu}
Để triển khai Collaborative Filtering, cần thực hiện các bước sau:

\textbf{Thu thập dữ liệu:} Thu thập dữ liệu đánh giá từ người dùng về các sản phẩm. Dữ liệu này có thể bao gồm các đánh giá sao, bình luận, và lịch sử mua hàng.

\textbf{Xây dựng ma trận người dùng-sản phẩm:} Tạo ra một ma trận mà mỗi hàng đại diện cho một người dùng và mỗi cột đại diện cho một sản phẩm. Các giá trị trong ma trận là các đánh giá mà người dùng đã cho sản phẩm đó.

\textbf{Tính toán độ tương tự:} Sử dụng các phương pháp như Cosine Similarity hoặc Pearson Correlation để tính toán độ tương tự giữa các người dùng hoặc các sản phẩm.

\textbf{Gợi ý sản phẩm:} Dựa trên độ tương tự đã tính toán, hệ thống gợi ý các sản phẩm mà người dùng tương tự đã đánh giá cao hoặc các sản phẩm tương tự với những gì người dùng đã thích.

\textbf{Đánh giá và cải thiện:} Đánh giá hiệu quả của hệ thống gợi ý bằng các chỉ số như RMSE (Root Mean Squared Error), MAE (Mean Absolute Error), hoặc Precision and Recall. Điều chỉnh các tham số và mô hình dựa trên kết quả đánh giá để cải thiện độ chính xác của hệ thống.

Collaborative Filtering là một phương pháp mạnh mẽ trong hệ thống gợi ý phim, giúp cá nhân hóa trải nghiệm người dùng dựa trên sự hợp tác và chia sẻ thông tin giữa các người dùng. Việc áp dụng đúng các kỹ thuật và thuật toán sẽ giúp cải thiện độ chính xác và hiệu quả của các đề xuất, từ đó tăng cường sự hài lòng của người dùng.

Chương này đã cung cấp một cái nhìn tổng quan và chi tiết về hai phương pháp gợi ý phim chủ đạo: Content-Based Filtering và Collaborative Filtering. Qua các phần trình bày về kiến thức nền tảng, cơ sở lý thuyết, các thuật toán và phương pháp nghiên cứu, chúng ta đã hiểu rõ cách thức hoạt động, ưu điểm và hạn chế của từng phương pháp.

Content-Based Filtering dựa vào phân tích các đặc trưng của nội dung phim để gợi ý các bộ phim tương tự mà người dùng có thể quan tâm. Phương pháp này giúp hệ thống đưa ra các gợi ý chính xác dựa trên thông tin cụ thể của phim và sở thích cá nhân của người dùng.

Collaborative Filtering dựa trên sự hợp tác và chia sẻ thông tin giữa các người dùng để gợi ý phim. Bằng cách phân tích hành vi và sở thích của nhiều người dùng, phương pháp này có thể đưa ra các gợi ý cá nhân hóa và phù hợp hơn, đặc biệt khi người dùng có xu hướng thích các phim mà những người dùng tương tự đã đánh giá cao.

Việc hiểu và áp dụng đúng các kỹ thuật của cả hai phương pháp này sẽ giúp hệ thống gợi ý phim hoạt động hiệu quả, mang lại trải nghiệm tốt hơn cho người dùng. Các kiến thức và phương pháp trình bày trong chương này sẽ là nền tảng vững chắc để phát triển và tối ưu hóa hệ thống gợi ý phim trong các chương tiếp theo.
\end{document}