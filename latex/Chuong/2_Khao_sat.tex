\documentclass[../DoAn.tex]{subfiles}
\begin{document}
Chương này tập trung vào việc khảo sát và phân tích các yêu cầu cần thiết để xây dựng ứng dụng gợi ý phim. Nội dung chương được tổ chức theo trình tự bắt đầu từ việc nghiên cứu và đánh giá các hệ thống gợi ý phim đang tồn tại trên thị trường, qua đó xác định được những điểm mạnh và điểm yếu của từng hệ thống. Dựa trên kết quả khảo sát, tôi sẽ xây dựng tổng quan về các chức năng cần thiết thông qua việc mô hình hóa bằng biểu đồ use case tổng quan, đồng thời tiến hành phân rã chi tiết các chức năng và các quy trình nghiệp vụ liên quan. Tiếp theo, mỗi chức năng sẽ được đặc tả một cách chi tiết và cụ thể, đảm bảo mọi yêu cầu đều được xác định đầy đủ và chính xác. Phần cuối của chương sẽ trình bày về các yêu cầu phi chức năng, trong đó tập trung vào chức năng gợi ý phim, nhằm đảm bảo hệ thống không chỉ thỏa mãn các yêu cầu chức năng cơ bản mà còn cung cấp trải nghiệm người dùng tối ưu nhất.

\section{Khảo sát hiện trạng}
\label{section:2.1}

\begin{figure}[!h]
\centering
\includegraphics[width=0.7\linewidth]{Hinhve/form_khao_sat_phim.png}
\caption{Form khảo sát người dùng về các ứng dụng}
\label{fig:Form_khao_sat_bieu_do}
\end{figure} 

\begin{figure}[!h]
\centering
\includegraphics[width=0.7\linewidth]{Hinhve/Khao_sat_bieu_do.png}
\caption{Khảo sát người dùng về các ứng dụng}
\label{fig:Khao_sat}
\end{figure} 

Để thực hiện khảo sát hiện trạng và xác định các yêu cầu cho ứng dụng gợi ý phim \autoref{fig:Form_khao_sat_bieu_do}, tôi đã tiến hành thu thập dữ liệu từ ba nguồn thông tin chính bao gồm: phản hồi từ người dùng và khách hàng, đánh giá các hệ thống hiện có trên thị trường, và nghiên cứu các ứng dụng tương tự. Dựa trên nguồn dữ liệu đã thu thập, tôi tiến hành phân tích một cách toàn diện, thực hiện so sánh giữa các giải pháp và đánh giá kỹ lưỡng các mặt ưu điểm cũng như hạn chế của từng sản phẩm và nghiên cứu liên quan.

Kết quả khảo sát từ phía người dùng và khách hàng \autoref{fig:Khao_sat} cho thấy, đối tượng này kỳ vọng một ứng dụng có thể thực hiện tìm kiếm phim với tốc độ cao và độ chính xác tốt, kèm theo đó là hệ thống gợi ý phim được cá nhân hóa dựa trên sở thích riêng và lịch sử xem phim của từng người. Bên cạnh đó, các chức năng như đánh giá phim, viết nhận xét và xem trailer cũng nhận được sự quan tâm đáng kể từ phía người dùng. Ngoài ra, việc nâng cao hiệu năng hệ thống và thiết kế giao diện thân thiện, dễ sử dụng cũng là những yêu cầu quan trọng được đề xuất.

\section{Khảo sát về quyền theo dõi}
\label{section:2.2_policy}

\begin{figure}[!h]
\centering
\includegraphics[width=0.7\linewidth]{Hinhve/form_khao_sat_theo_doi.png}
\caption{Form khảo sát về quyền theo dõi}
\label{fig:Form_khao_sat_policy}
\end{figure}

\begin{figure}[!h]
\centering
\includegraphics[width=0.7\linewidth]{Hinhve/Khao_sat_policy.png}
\caption{Khảo sát về quyền theo dõi}
\label{fig:Khao_sat_policy}
\end{figure} 

Theo kết quả của khảo sát về quyền theo dõi \autoref{fig:Khao_sat_policy}, tôi nhận thấy rằng người dùng mong muốn một ứng dụng gợi ý phim, nhưng phần lớn lại không mong muốn bị theo dõi hành vi trên ứng dụng. Vì vậy, tôi sẽ phát triển thêm chức năng theo dõi hành vi người dùng để gợi ý phim một cách phù hợp nhất với mong muốn người dùng.

\begin{table}[!h]
\centering
\caption{Khảo sát dựa trên các hệ thống giới thiệu phim hiện có}
\includegraphics[width=1\linewidth]{Hinhve/Khao_sat_cac_web.png}
\label{tab:Khao_sat_web}
\end{table} 

% \begin{tabular}{|c|c|c|c|c|c|c|} 
%  \hline
%  Tính năng & IMDb & Meta Critic & Rotten Tomatoes & TMDb & Box Office Mojo & The Numbers \\
%  \hline
%  Thông tin phim & Đầy đủ & Đầy đủ & Đầy đủ & Đầy đủ nhưng phải lấy bằng API & Cơ bản & Thiếu một số \\ 
%  \hline
%  Gợi ý phim & Có & Không & Không & Không & Không & Không \\
%  \hline
%  Review & Có & Có & Có & Có & Không & Không \\
%  \hline
%  Trailer & Có & Có & Có & Có & Không & Không \\
%  \hline
%  Hiệu năng & Chậm & Vừa phải & Vừa phải & Ổn & Vừa phải & Ổn \\ 
%  \hline
% \end{tabular}

Mặt khác, việc nghiên cứu và đánh giá các hệ thống hiện có bao gồm IMDb, Metacritic, Rotten Tomatoes, TMDb, Box Office Mojo và The Numbers đã được thực hiện nhằm so sánh và phân tích các tính năng, hiệu suất hoạt động cũng như khả năng cung cấp thông tin của mỗi hệ thống. Kết quả phân tích cho thấy rằng mỗi hệ thống đều sở hữu những ưu điểm và nhược điểm đặc trưng, và việc tích hợp các tính năng ưu việt từ các hệ thống này có tiềm năng tạo ra một giải pháp gợi ý phim mang tính toàn diện và hiệu quả.

Bên cạnh đó, một số ứng dụng giới thiệu phim trên các nền tảng di động và web cũng đã được khảo sát, qua đó phát hiện ra những đặc điểm tương đồng liên quan đến thiết kế giao diện thân thiện với người dùng và các tính năng gợi ý phim được xây dựng dựa trên phân tích hành vi người dùng như được thể hiện trong \autoref{tab:Khao_sat_web}. Tuy vậy, phần lớn các ứng dụng này vẫn còn tồn tại những hạn chế về khả năng xử lý hiệu suất và sự đa dạng trong các tính năng được cung cấp.

Tổng hợp từ các kết quả khảo sát đã thu thập được, các tính năng quan trọng cần được tích hợp vào hệ thống đã được xác định một cách rõ ràng, nhằm mục tiêu phát triển một giải pháp gợi ý phim toàn diện có khả năng đáp ứng một cách hiệu quả các nhu cầu đa dạng của người dùng.

\section{Tổng quan chức năng}
\label{section:2.2}
\subsection{Biểu đồ use case tổng quát}
\label{subsection:2.2.1}

\textbf{Các tác nhân tham gia:}

\begin{itemize}
\item Guest (Khách truy cập): Đối tượng chưa đăng ký thành viên, thực hiện truy cập ứng dụng nhằm mục đích tra cứu và tham khảo các dữ liệu liên quan đến phim ảnh.
\item User (Thành viên): Những cá nhân đã xác thực tài khoản, được phép thực hiện các thao tác như gửi phản hồi về phim, quản lý danh mục phim ưa thích, cập nhật hồ sơ cá nhân và sử dụng đầy đủ các tiện ích mở rộng khác.
\item Admin (Quản trị viên): Cá nhân nắm quyền điều hành cao nhất, thực hiện giám sát toàn diện và kiểm soát mọi dữ liệu cũng như hoạt động diễn ra trên nền tảng.
\end{itemize}

\textbf{Vai Trò của Mỗi Tác Nhân:}

\begin{itemize} 
  \item Guest: Vai trò chủ đạo của đối tượng này là tra cứu dữ liệu điện ảnh và thưởng thức các tác phẩm trực tuyến mà không yêu cầu thực hiện thao tác đăng nhập. 
  \item User: Ngoài việc sở hữu các tính năng như Guest, thành viên này còn được cấp quyền bình luận, lưu trữ phim vào mục quan tâm và điều chỉnh hồ sơ cá nhân. 
  \item Admin: Đây là chủ thể điều hành toàn bộ hệ thống, chịu trách nhiệm kiểm soát danh mục phim cũng như giám sát các hoạt động của người dùng. 
\end{itemize}

\textbf{Các use case chính:}

\begin{itemize} 
  \item Truy xuất dữ liệu điện ảnh: Cả hai đối tượng Guest và User đều được phép tham khảo các thông số chi tiết của tác phẩm như cốt truyện, ê-kíp sản xuất, dàn sao, giải thưởng, chỉ số doanh thu và những nhận xét từ cộng đồng. 
  \item Phản hồi và lưu trữ: Thành viên hệ thống có khả năng trực tiếp chấm điểm cho tác phẩm cũng như cá nhân hóa kho lưu trữ bằng cách thêm phim vào mục yêu thích. 
  \item Đề xuất nội dung: Dựa trên lịch sử truy cập và các lượt xếp hạng từ phía người dùng, nền tảng sẽ tự động hiển thị danh sách các bộ phim phù hợp với sở thích cá nhân. 
\end{itemize}


\begin{figure}[H]
\centering
\includegraphics[width=0.75\linewidth]{Hinhve/Usecase_tong_quan.png}
\caption{Use case tổng quan}
\label{fig:Usecase_tong_quan}
\end{figure} 

\subsection{Biểu đồ use case phân rã Quản lý thông tin người dùng}
\label{subsection:2.2.2}
\begin{figure}[H]
\centering
\includegraphics[width=0.6\linewidth]{Hinhve/Usecase_phan_ra_admin_quanlynguoidung.drawio.png}
\caption{Phân rã use case Quản lý thông tin người dùng}
\label{fig:Usecase_quanlynguoidung}
\end{figure} 
Trong sơ đồ use case phân rã "Quản lý thông tin người dùng", các thành phần con được chia tách và giải thích cụ thể như sau:

\begin{itemize}
\item Thêm người dùng: Admin có quyền thêm người dùng mới vào hệ thống bằng cách cung cấp thông tin cần thiết như tên đăng nhập, email và mật khẩu.
\item Chỉnh sửa hồ sơ thành viên: Quản trị viên được quyền thay đổi các chi tiết cá nhân của người dùng, bao gồm tên truy cập, địa chỉ email cũng như mật khẩu hệ thống.
\item Loại bỏ thành viên: Quản trị viên có quyền hạn gỡ bỏ tài khoản người dùng ra khỏi hệ thống.
\end{itemize}

\subsection{Biểu đồ use case phân rã Quản lý thông tin phim}
\label{subsection:2.2.3}
\begin{figure}[H]
\centering
\includegraphics[width=0.6\linewidth]{Hinhve/Usecase_phan_ra_admin_quanlyphim.drawio.png}
\caption{Phân rã use case Quản lý thông tin phim}
\label{fig:Usecase_quanlyphim}
\end{figure} 
Trong sơ đồ Use Case chi tiết thuộc phân hệ "Quản lý thông tin phim", các chức năng con được phân tách và diễn giải cụ thể như sau:

\begin{itemize} 
  \item Cập nhật phim mới: Quản trị viên có quyền thiết lập các bộ phim mới trên hệ thống thông qua việc cung cấp đầy đủ các dữ liệu như tên phim, tóm tắt, chủng loại, đạo diễn, dàn diễn viên, thời điểm phát hành, ảnh minh họa, ... 
  \item Chỉnh sửa dữ liệu điện ảnh: Admin được phép thay đổi những thông số chi tiết của tác phẩm đã có sẵn, bao gồm các thành phần về thể loại, người dàn dựng, diễn viên tham gia và mô tả kịch bản. 
  \item Gỡ bỏ phim: Người quản lý hệ thống có thẩm quyền xóa vĩnh viễn một đầu phim ra khỏi cơ sở dữ liệu. 
\end{itemize}


\subsection{Biểu đồ use case phân rã Quản lý tài khoản}
\label{subsection:2.2.4}
\begin{figure}[H]
\centering
\includegraphics[width=0.6\linewidth]{Hinhve/Usecase_phan_ra_user_quanlytaikhoan.drawio.png}
\caption{Phân rã use case Quản lý tài khoản}
\label{fig:Usecase_quanlytaikhoan}
\end{figure} 
Trong biểu đồ Use case phân rã "Quản lý tài khoản", các chức năng con được phân tách và diễn giải cụ thể như sau:

\begin{itemize} 
  \item Quên mật khẩu: Thành viên có thể thiết lập lại mã bảo mật thông qua việc yêu cầu hệ thống gửi đường dẫn thay đổi mật khẩu tới hòm thư điện tử cá nhân. 
  \item Đổi mật khẩu: Người dùng được phép thay thế mật khẩu đang sử dụng bằng cách xác nhận lại mật khẩu cũ và thiết lập chuỗi ký tự mới. 
  \item Thay đổi hồ sơ cá nhân: Người dùng có quyền tự điều chỉnh các chi tiết về bản thân như họ tên, địa chỉ email, hình ảnh đại diện và các thông tin liên quan khác. 
\end{itemize}


\subsection{Biểu đồ use case phân rã Tương tác với phim}
\label{subsection:2.2.5}
\begin{figure}[H]
\centering
\includegraphics[width=0.6\linewidth]{Hinhve/Usecase_phan_ra_user_tuongtacphim.drawio.png}
\caption{Phân rã use case Tương tác với phim}
\label{fig:Usecase_tuongtacphim}
\end{figure} 
Trong biểu đồ use case phân rã "Tương tác với phim", các chức năng con được phân tách và diễn giải cụ thể như sau:

\begin{itemize} 
  \item Lưu phim vào mục yêu thích: Thành viên có thể thêm các tác phẩm điện ảnh vào bộ sưu tập cá nhân nhằm thuận tiện cho việc tìm kiếm lại, đồng thời giúp hệ thống đề xuất các nội dung tương đồng phù hợp với sở thích. 
  \item Loại bỏ phim khỏi danh sách: Người dùng được phép gỡ bỏ những bộ phim đã lưu trước đó ra khỏi mục yêu thích của mình. 
\end{itemize} 

\subsection{Biểu đồ use case phân rã bình luận, đánh giá phim}
\label{subsection:2.2.6}
\begin{figure}[H]
\centering
\includegraphics[width=0.6\linewidth]{Hinhve/Usecase_phan_ra_user_review.drawio.png}
\caption{Phân rã use case bình luận, đánh giá phim}
\label{fig:Usecase_review}
\end{figure} 
Trong biểu đồ use case phân rã "bình luận, đánh giá phim", các chức năng con được phân tách và diễn giải cụ thể như sau:

\begin{itemize} 
  \item Đánh giá phim: Sau khi thưởng thức, thành viên có thể soạn thảo và chia sẻ nhận xét về phim lên hệ thống để thảo luận cũng như đóng góp quan điểm cá nhân với cộng đồng. 
  \item Tương tác bình luận: Người dùng có quyền biểu đạt sự tán thành hoặc khích lệ bằng cách nhấn nút "Thích" cho những bài review mà họ cảm thấy tâm đắc hoặc có giá trị tham khảo cao. 
\end{itemize}


% \subsection{Quy trình nghiệp vụ}
% \label{subsection:2.2.7}
% Nếu sản phẩm/hệ thống cần xây dựng có quy trình nghiệp vụ quan trọng/đáng chú ý, sinh viên cần mô tả và vẽ biểu đồ hoạt động minh họa quy trình nghiệp vụ đó. Sinh viên lưu ý đây không phải là luồng sự kiện của từng use case, mà là luồng hoạt động kết hợp nhiều use case để thực hiện một nghiệp vụ nào đó.

% Ví dụ, một hệ thống quản lý thư viện có quy trình nghiệp vụ mượn trả với mô tả sơ bộ như sau: Sinh viên làm thẻ mượn, sau đó sinh viên đăng ký mượn sách, thủ thư cho mượn, và cuối cùng sinh viên trả lại sách cho thư viện. Một hệ thống có thể có một vài quy trình nghiệp vụ quan trọng như vậy.
\section{Đặc tả chức năng}
\label{section:2.3}
\subsection{Đặc tả use case Đăng nhập}
\begin{table}[H]
\centering
\caption{Đặc tả use case Đăng nhập}
\includegraphics[width=0.9\linewidth]{Hinhve/Dac_ta_dangnhap.png}
\label{tab:Usecase_dangnhap}
\end{table} 

Bảng trên (\autoref{tab:Usecase_dangnhap}) là đặc tả use case Đăng nhập từ người dùng. Trong đó, người dùng có thể đăng nhập vào hệ thống bằng cách nhập tên đăng nhập và mật khẩu. Hệ thống sẽ kiểm tra tên đăng nhập và mật khẩu của người dùng và trả về kết quả đăng nhập. Sẽ validate tên đăng nhập và mật khẩu để tránh lỗi input không hợp lệ.

\subsection{Đặc tả use case Chỉnh sửa thông tin cá nhân}
\begin{table}[H]
\centering
\caption{Đặc tả use case Chỉnh sửa thông tin cá nhân}
\includegraphics[width=0.9\linewidth]{Hinhve/Dac_ta_chinhsuathongtincanhan.png}
\label{tab:Usecase_chinhsuathongtincanhan}
\end{table} 

Bảng \autoref{tab:Usecase_chinhsuathongtincanhan} mô tả chi tiết use case Chỉnh sửa thông tin cá nhân được thực hiện bởi người dùng. Use case này cho phép người dùng thực hiện việc cập nhật các thông tin cá nhân bao gồm tên, địa chỉ email, ảnh đại diện và các thông tin khác. Hệ thống sẽ tiến hành xác thực và kiểm tra tính hợp lệ của các thông tin được cập nhật, sau đó phản hồi lại kết quả của quá trình chỉnh sửa. Việc kiểm tra tính hợp lệ của dữ liệu đầu vào được thực hiện nhằm ngăn chặn các lỗi do dữ liệu không đúng định dạng hoặc không đáp ứng các yêu cầu đã quy định.

\subsection{Đặc tả use case Tìm kiếm phim}
\begin{table}[H]
\centering
\caption{Đặc tả use case Tìm kiếm phim}
\includegraphics[width=0.9\linewidth]{Hinhve/Dac_ta_timkiemphim.png}
\label{tab:Usecase_timkiemphim}
\end{table} 

Bảng trên (\autoref{tab:Usecase_timkiemphim}) là đặc tả use case Tìm kiếm phim từ người dùng. Trong đó, người dùng có thể tìm kiếm phim bằng cách nhập tên phim, tên diễn viên, tên đạo diễn, ... Hệ thống sẽ kiểm tra tên phim, tên diễn viên, tên đạo diễn của người dùng và trả về kết quả tìm kiếm. Sẽ validate tên phim, tên diễn viên, tên đạo diễn để tránh lỗi input không hợp lệ.

Ngoài ra còn có chức năng Tìm kiếm phim bằng ngôn ngữ người dùng, khi người dùng nhập vào ngôn ngữ người dùng, hệ thống sẽ phân tích nội dung người dùng nhập vào và trả về kết quả tìm kiếm.

\subsection{Đặc tả use case Đánh dấu phim mình thích}
\begin{table}[H]
\centering
\caption{Đặc tả use case Đánh dấu phim mình thích}
\includegraphics[width=0.9\linewidth]{Hinhve/Dac_ta_themvaolistyeuthich.png}
\label{tab:Usecase_danhdayphimyeuthich}
\end{table} 

Bảng \autoref{tab:Usecase_danhdayphimyeuthich} mô tả chi tiết use case Đánh dấu phim yêu thích được thực hiện bởi người dùng. Thông qua use case này, người dùng có khả năng lưu các bộ phim yêu thích vào một danh sách cá nhân riêng biệt, tạo điều kiện thuận lợi cho việc quản lý và truy cập lại sau này. Đồng thời, hành động này cũng cung cấp dữ liệu quan trọng để hệ thống có thể phân tích và đề xuất các bộ phim phù hợp với sở thích của người dùng.

\subsection{Đặc tả use case Đánh giá phim}
\FloatBarrier
\begin{table}[H]
\centering
\caption{Đặc tả use case Đánh giá phim}
\includegraphics[width=0.9\linewidth]{Hinhve/Dac_ta_danhgiaphim.png}
\label{tab:Usecase_danhgiaphim}
\end{table} 
\FloatBarrier

Bảng trên (\autoref{tab:Usecase_danhgiaphim}) là đặc tả use case Đánh giá phim từ người dùng. Trong đó, người dùng có thể đánh giá sao, để lại bình luận sau khi xem. Hệ thống sẽ kiểm tra đánh giá của người dùng và trả về kết quả đánh giá. Sẽ validate đánh giá và bình luận để tránh lỗi input không hợp lệ.

\FloatBarrier
\subsection{Đặc tả use case quản lý phim}
\begin{table}[H]
\centering
\caption{Đặc tả use case quản lý phim}
\includegraphics[width=0.9\linewidth]{Hinhve/Dac_ta_crudphim.png}
\label{tab:Usecase_crudphim}
\end{table}
\FloatBarrier

Bảng \autoref{tab:Usecase_crudphim} trình bày chi tiết use case CRUD phim dành cho quản trị viên. Use case này cho phép quản trị viên thực hiện các thao tác quản lý cơ bản đối với danh sách phim trong hệ thống, bao gồm việc thêm mới, chỉnh sửa và xóa bỏ các bộ phim. Sau khi nhận được yêu cầu từ quản trị viên, hệ thống sẽ tiến hành xác thực và xử lý dữ liệu, sau đó phản hồi lại kết quả của các thao tác CRUD đã được thực hiện.

\subsection{Đặc tả use case Gợi ý phim theo nội dung}
\begin{table}[H]
\centering
\caption{Đặc tả use case Gợi ý phim theo nội dung}
\includegraphics[width=0.9\linewidth]{Hinhve/Dac_ta_goiycontentbased.png}
\label{tab:Usecase_goiycontentbased}
\end{table} 

Bảng trên (\autoref{tab:Usecase_goiycontentbased}) là đặc tả use case Gợi ý phim theo nội dung từ người dùng. Trong đó, ứng dụng dựa vào tên phim, thể loại phim, diễn viên, đạo diễn, ... để gợi ý phim tương đồng với phim đang xem.

\section{Yêu cầu phi chức năng}
\label{section:2.4}
Phần này trình bày các yêu cầu phi chức năng và các yêu cầu kỹ thuật mà hệ thống cần đáp ứng, được xác định như sau:

Hiệu năng: Hệ thống phải đảm bảo đạt được hiệu suất hoạt động cao và có khả năng phản hồi nhanh chóng. Cụ thể, đối với các thao tác CRUD cơ bản, thời gian phản hồi yêu cầu phải nhỏ hơn 3 giây, trong khi đối với các chức năng tìm kiếm và gợi ý phim sử dụng công nghệ AI, thời gian phản hồi có thể lên đến 10 giây. Ngoài ra, việc tối ưu hóa thời gian tải trang là một yêu cầu quan trọng nhằm nâng cao chất lượng trải nghiệm của người dùng.

Tính dễ sử dụng: Giao diện người dùng cần được xây dựng với nguyên tắc đơn giản, trực quan và dễ hiểu, tạo điều kiện thuận lợi cho người dùng trong việc tìm kiếm thông tin và thực hiện các tương tác với hệ thống một cách tự nhiên và thuận tiện nhất.

Tính dễ bảo trì: Mã nguồn của ứng dụng cần được tổ chức theo cấu trúc rõ ràng, logic và dễ dàng bảo trì, tạo điều kiện thuận lợi cho việc thực hiện các công việc nâng cấp hệ thống, khắc phục lỗi và mở rộng các tính năng mới trong tương lai một cách hiệu quả.

Cơ sở dữ liệu: Hệ thống cần được tích hợp với một hệ quản trị cơ sở dữ liệu phù hợp để thực hiện việc lưu trữ và quản lý dữ liệu về phim cũng như thông tin người dùng một cách hiệu quả và linh hoạt. Bên cạnh đó, hệ thống cần đảm bảo khả năng thực hiện các truy vấn dữ liệu với tốc độ cao để đáp ứng yêu cầu về hiệu năng.

%%%%%%%%%%%%%%%%%%%%%%%%%%%%%%%%%%%

\end{document}