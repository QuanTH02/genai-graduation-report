\documentclass[../DoAn.tex]{subfiles}
\begin{document}
Chương này tôi sẽ giới thiệu về quá trình thực hiện và phát triển ứng dụng gợi ý phim bằng genAI. Tôi sẽ trình bày các bước và quy trình đã thực hiện để xây dựng sản phẩm từ các ý tưởng ban đầu đến việc triển khai và thử nghiệm trên môi trường local. Bên cạnh đó, chương cũng nhấn mạnh các mục tiêu và những thách thức trong quá trình xây dựng sản phẩm này, cùng với những kết quả đạt được và học hỏi được từ dự án.

\textbf{Kết quả so sánh}
\begin{table}[H]
\centering
\includegraphics[width=1\linewidth]{Hinhve/sosanh.png}
\caption{Kết quả so sánh so với các ứng dụng khác}
\label{tab:sosanh}
\end{table} 

Nhìn vào \autoref{tab:sosanh} có thể thấy ứng dụng hiện tại đã giải quyết được những vấn đề ban đầu đặt ra.

\textbf{Phân tích quá trình thực hiện}

Trong suốt quá trình thực hiện ĐATN, tôi đã hoàn thành được những mục tiêu chính sau:
\begin{enumerate}
    \item Lấy được tất cả những dữ liệu cần thiết để phục vụ cho việc xây dựng ứng dụng. Để làm được mục tiêu này, tôi đã tìm rất nhiều tài liệu riêng về chủ đề crawl để có thể nghiên cứu và áp dụng được hoàn toàn vào dự án của bản thân.
    \item Đã hoàn thành chức năng chính của ứng dụng. Tôi đã thêm được rất nhiều chức năng mà bản thân đã nghĩ khó có thể làm, nhờ đó tôi có thể biết được nhiều công nghệ, kiến thức hay ho hơn để áp dụng vào những dự án trong tương lai.
    \item Đã triển khai và thử nghiệm ứng dụng trên môi trường local.
\end{enumerate}

\textbf{Những điểm chưa làm được}
\begin{enumerate}
    \item Chưa triển khai, deploy lên server thực tế. Hiện tại ứng dụng chỉ chạy trên môi trường local, chưa được kiểm thử với lượng người dùng lớn trong điều kiện tải cao. 
    \item Tổng quan ứng dụng đang còn đơn giản. Nhìn qua, ứng dụng có giao diện, luồng thao tác còn đang đơn giản, hầu như chỉ view.
\end{enumerate}

\textbf{Đóng góp nổi bật}
\begin{enumerate}
    \item Thiết kế và triển khai ứng dụng được thiết kế ngay từ đầu: Tôi đã hoàn thành các chức năng cơ bản và đảm bảo ứng dụng hoạt động ổn định lâu dài.
    \item Dữ liệu đảm bảo chính xác, nhất quán, đầy đủ và cập nhật liên tục. Tôi đã crawl được bộ dữ liệu hoàn chỉnh phục vụ cho mục đích thử nghiệm, xây dựng ứng dụng của bản thân.
    \item Xây dựng được chức năng gợi ý phim theo lịch sử hoạt động cũng như tìm kiếm phim bằng ngôn ngữ tự nhiên của người dùng.
    \item Cấu trúc mã nguồn của tôi cho phép mở rộng và có thể phát triển các tính năng nâng cao, dễ dàng kế thừa và phát triển tiếp.
\end{enumerate}

\textbf{Bài học kinh nghiệm cho bản thân}
Trong quá trình thực hiện dự án, tôi đã rút ra được rất nhiều bài học như:
\begin{enumerate}
    \item Cách quản lý thời gian: Việc lập kế hoạch chi tiết và quản lý thời gian hiệu quả giúp hoàn thành các mục tiêu đúng tiến độ. Vì hiểu được điều này, các buổi báo cáo với cô giáo diễn ra rất suôn sẻ và không bị chậm trễ.
    \item Khả năng học hỏi và tự nghiên cứu: Khả năng tự học là rất quan trọng, bởi trên nhà trường, chỉ học lý thuyết và nhiệm vụ của chúng tôi là về nhà tìm tòi, mày mò thực hành thêm để ghi nhớ những kiến thức đó. Chỉ có tự ôn tập, nghiên cứu mới có thể khiến bản thân tiến bộ hơn.
    \item Thu thập thông tin: Tôi rút ra được bài học rằng, đây là bước rất quan trọng trước khi xây dựng một ứng dụng. Chúng ta phải biết người dùng muốn gì, chúng ta mới xây dựng được một ứng dụng thành công.
\end{enumerate}

Hiện tại, tôi nghĩ các công việc cần thiết để hoàn thiện sản phẩm của mình là:
\begin{enumerate}
    \item Đầu tiên là triển khai trên môi trường thực tế. Ứng dụng của tôi hiện tại đã phân chia ra đúng các gói, các lớp, các thành phần và đó là cơ sở để có thể dễ dàng deploy trong tương lai. Đây là một nhiệm vụ cần thiết để có thể phổ biến được ứng dụng của bản thân.
    \item Tôi cần cải thiện về giao diện ứng dụng. Ứng dụng của tôi hiện tại đang có những tông màu rất đơn giản như đen, xanh, xanh lá cây. Để ứng dụng bắt mắt, được sự chú ý của người khác hơn thì tôi nghĩ nên cải thiện nhiều hơn về giao diện ứng dụng.
    \item Tích hợp thêm tính năng phân tích và thống kê: Hiện tại ứng dụng chỉ có thống kê số lượng đánh giá. Mà người dùng thì lại rất quan tâm đến số liệu để xem phim theo số đông, nêm tôi nghĩ đây là bước cần thiết hiện tại.
\end{enumerate}

Trong tương lai, tôi muốn phát triển thêm vài chức năng khác như:
\begin{enumerate}
    \item Phát triển ứng dụng trên thiết bị di động. Điện thoại di động là thiết bị không thể thiếu đối với mọi người, vậy nên phát triển ứng dụng trên thiết bị di động trong tương lại là một việc rất quan trọng.
    \item Tích hợp thêm hệ thống dự đoán doanh thu phim mới. Đây là một chức năng mà tôi đã định hướng từ trước, nhưng do dữ liệu tôi triển khai trên hệ thống là không đủ, nên tôi muốn trong tương lai, tôi có thể lấy nhiều dữ liệu hơn và phát triển tính năng này
\end{enumerate}

Trên đây là kết luận và hướng phát triển trong tương lai của em. Tôi sẽ luôn cố gắng hết sức mình để hoàn thiện sản phẩm của mình một cách tốt nhất, tôi xin chân thành cảm ơn!
\end{document}