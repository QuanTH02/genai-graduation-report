\documentclass[../DoAn.tex]{subfiles}
\begin{document}

\section{Thiết kế kiến trúc}
\subsection{Lựa chọn kiến trúc phần mềm}
Trong sản phẩm này, em lựa chọn kiến trúc phần mềm MVT trong Django.

\textbf{Giải thích sơ bộ về MVT:}

Kiến trúc MVT của Django tương tự như kiến trúc MVC (Model-View-Controller), nhưng có sự khác biệt về cách hoạt động.

\begin{itemize}
    \item \textbf{Model (M):} Định nghĩa cấu trúc dữ liệu của ứng dụng và quản lý truy cập vào cơ sở dữ liệu.
    \item \textbf{View (V):} Xử lý logic xử lý nghiệp vụ và hiển thị dữ liệu cho người dùng.
    \item \textbf{Template (T):} Định nghĩa giao diện người dùng bằng cách sử dụng HTML cùng với các biến Python.
\end{itemize}

\textbf{Mô tả kiến trúc cụ thể cho ứng dụng:}

\begin{itemize}
    \item \textbf{Model (M):} 
    
Trong ứng dụng của em, các model như MovieInformation, Awards, Director, Cast, Genres, và các model liên quan khác sẽ đại diện cho các đối tượng dữ liệu chính về phim, đạo diễn, diễn viên, thể loại, v.v.

Mỗi model sẽ có các trường (fields) tương ứng và quan hệ (relationships) với các model khác để lưu trữ và quản lý thông tin một cách logic và cấu trúc hóa.
    
    \item \textbf{View (V):} 

Các view như FilmListView, DirectorListView, GenreListView, ReviewView, và các view khác trong danh sách của em sẽ đảm nhiệm xử lý các yêu cầu từ người dùng.

View sẽ truy xuất dữ liệu từ model thông qua các queryset, xử lý logic nghiệp vụ và chuẩn bị dữ liệu để truyền cho template.

    \item \textbf{Template (T):} 

Sử dụng các template HTML của Django kết hợp với template tags hiển thị dữ liệu cho người dùng.

Template sẽ nhận dữ liệu từ view và render ra các thành phần giao diện như danh sách phim, thông tin chi tiết, biểu đồ, v.v.

\end{itemize}

\subsection{Thiết kế tổng quan}
\textbf{Các gói và sự phụ thuộc giữa các gói: }

\begin{figure}[H]
    \centering
    \includegraphics[width=1\linewidth]{Hinhve/imgUML.png}
    \caption{Biểu đồ phụ thuộc gói}
    \label{fig:imgUML}
\end{figure}

\begin{enumerate}
    \item Data Access Layer (Model)

    \textbf{Mục đích:} Chứa các định nghĩa về dữ liệu và cấu trúc dữ liệu của ứng dụng.
    
    \textbf{Nhiệm vụ:} Mô hình hóa các thực thể trong hệ thống, bao gồm các bảng cơ sở dữ liệu và các mối quan hệ giữa chúng.
    
    \textbf{Các thành phần:} MovieInformation, Awards, Director, Cast, Genres, v.v.
    
    \textbf{Phụ thuộc:} Không phụ thuộc vào bất kỳ gói nào khác. Đây là tầng thấp nhất trong kiến trúc, nơi dữ liệu được định nghĩa và quản lý.

    \item Business Logic Layer (View)

    \textbf{Mục đích:} Xử lý logic nghiệp vụ và quản lý các yêu cầu từ người dùng.
    
    \textbf{Nhiệm vụ:} Truy xuất dữ liệu từ các model, xử lý và chuẩn bị dữ liệu để hiển thị trong các template.
    
    \textbf{Các thành phần:} FilmListView, DirectorListView, GenreListView, ReviewView, v.v.
    
    \textbf{Phụ thuộc:} Phụ thuộc vào gói Models. View sẽ truy xuất dữ liệu từ model và xử lý logic nghiệp vụ. Không phụ thuộc vào gói Templates trực tiếp. Dữ liệu được chuẩn bị trong view sẽ được truyền đến template để hiển thị.

    \item Presentation Layer (Template)

    \textbf{Mục đích:} Định nghĩa giao diện người dùng.
    
    \textbf{Nhiệm vụ:} Hiển thị dữ liệu được truyền từ view dưới dạng HTML, CSS và JavaScript.
    
    \textbf{Các thành phần:} Các tệp HTML template cho các trang web khác nhau của ứng dụng.
    
    \textbf{Phụ thuộc:} Phụ thuộc vào gói Views. Templates nhận dữ liệu từ view và hiển thị nó. Không phụ thuộc vào gói Models trực tiếp. Mọi dữ liệu hiển thị trên giao diện đều được truyền từ view.

\end{enumerate}

\subsection{Thiết kế chi tiết gói}

\begin{figure}[H]
    \centering
    \includegraphics{Hinhve/layerModel.png}
    \caption{Thiết kế chi tiết gói ở tầng Model}
    \label{fig:layerModel}
\end{figure}

\textbf{Gói Model:}

Ở gói này, có các gói nhỏ hợp thành là User và Movie. 

\begin{itemize}
    \item \textbf{Gói User:} 
    Gồm 3 class l là Review, User và LikeMovie. 

    Mô tả: Class User: Đại diện cho thông tin người dùng. Class Review: Chứa thông tin về đánh giá phim của người dùng. Class LikeMovie: Lưu trữ những bộ phim mà người dùng thích.
    
    Phụ thuộc: Class Review và class LikeMovie phụ thuộc vào thông tin của User. 
    
    \item  \textbf{Gói Movie:} 
    
    Gồm 5 class là Movieinfomation, Director, Award, Cast, Writer. 

    Mô tả: Class MovieInformation: Lớp chính quản lý thông tin tổng quát về phim. Class Director: Lưu trữ thông tin chi tiết về các đạo diễn của phim. Class Award: Lưu trữ thông tin về các giải thưởng phim đã nhận được. Class Cast: Lưu trữ thông tin về các diễn viên tham gia trong phim. Class Writer: Lưu trữ thông tin về các nhà biên kịch của phim.

    Phụ thuộc: 4 class còn lại có quan hệ phụ thuộc với Movieinfomation, lấy Movieinfomation làm gốc, các thông tin còn lại đều là thông tin bổ sung.
\end{itemize}

\begin{figure}[H]
    \centering
    \includegraphics{Hinhve/layerView.png}
    \caption{Thiết kế chi tiết gói ở tầng View}
    \label{fig:layerView}
\end{figure}

\textbf{Gói View:}

Ở gói này, có các gói nhỏ hợp thành là UserView, MovieView và AdminView.

\begin{itemize}
    \item \textbf{Gói UserView:} 
    Gồm 3 class l là ReviewView, UserView và LikeMovieView. 

    Mô tả: Class UserView: Lớp này chịu trách nhiệm hiển thị và xử lý các thông tin liên quan đến người dùng, bao gồm đăng ký, đăng nhập, và thông tin cá nhân. Class ReviewView: Lớp này xử lý và hiển thị các đánh giá phim của người dùng. Nó phụ thuộc vào thông tin người dùng từ UserView. Class LikeMovieView: Lớp này quản lý và hiển thị các bộ phim mà người dùng đã thích. Nó cũng phụ thuộc vào thông tin người dùng từ UserView.
    
    Phụ thuộc: Class ReviewView và class LikeMovieView phụ thuộc vào thông tin của UserView. 

    \item  \textbf{Gói MovieView:} 
    
    Gồm 5 class là MovieinfomationView, DirectorView, AwardView, CastView, WriterView. 

    Mô tả: Class MovieInformationView: Lớp chính quản lý, thao tác chính về phim. Class DirectorView: Xử lý dữ liệu về đạo diễn. Class AwardView: Xử lý dữ liệu, các thao tác về giải thưởng. Class CastView: Xử lý dữ liệu, các thao tác tính về diễn viên. Class Writer: Thao tác dữ liệu với biên kịch.

    Phụ thuộc: 4 class còn lại có quan hệ phụ thuộc với Movieinfomation, lấy Movieinfomation làm gốc, các thông tin còn lại đều là thông tin bổ sung.

    \item  \textbf{Gói AdminView:} 
    
    Gồm 1 class là AdminView. 

    Mô tả: Class AdminView có thể xử lý toàn bộ yêu cầu bao gồm các yêu cầu giống người dùng và có thể khác, admin quản lý toàn bộ trang web. Có thể thêm, sửa, xóa toàn bộ dữ liệu.

    Phụ thuộc: Không phụ thuộc vào class nào.
\end{itemize}

\begin{figure}[H]
    \centering
    \includegraphics[width=1\linewidth]{Hinhve/layerTemp.png}
    \caption{Thiết kế chi tiết gói ở tầng Template}
    \label{fig:layerTemp}
\end{figure}

\textbf{Gói Template:}

Gồm 2 class l là UserTemplate, AdminTemplate.

Mô tả: Class UserTemplate: Lớp này chịu trách nhiệm quản lý các mẫu HTML dành cho giao diện người dùng. Các mẫu này bao gồm các trang đăng nhập, đăng ký, trang cá nhân, trang đánh giá phim, và các trang hiển thị danh sách phim, diễn viên, đạo diễn, v.v. Class AdminTemplate: Lớp này chịu trách nhiệm quản lý các mẫu HTML dành cho giao diện admin. Các mẫu này bao gồm các trang quản lý người dùng, quản lý phim, duyệt đánh giá, và các trang khác mà admin cần sử dụng để quản lý hệ thống.
    
Phụ thuộc: 2 class này không có mối quan hệ phụ thuộc với nhau. Chúng độc lập và phục vụ các mục đích khác nhau: một bên là giao diện người dùng và một bên là giao diện admin.

\section{Thiết kế chi tiết}
\subsection{Thiết kế giao diện}
Khi thiết kế giao diện cho ứng dụng, em đã chọn các thông số và tiêu chuẩn nhất định để đảm bảo tính nhất quán và tiện dụng. Các màn hình mục tiêu cho ứng dụng này có độ phân giải 1920x1080 pixels, phù hợp với các kích thước màn hình từ 13 inch trở lên. Ứng dụng sẽ hỗ trợ hiển thị 16 triệu màu để đảm bảo các yếu tố đồ họa và hình ảnh hiển thị sắc nét.

Trong quá trình thiết kế, em đã áp dụng các quy chuẩn cụ thể để tạo ra giao diện người dùng nhất quán. Các nút bấm được thiết kế với các kích thước chuẩn, dễ dàng nhận biết và sử dụng. Các điều khiển như dropdowns, checkboxes, và radio buttons được thiết kế để có giao diện đồng nhất và rõ ràng. Các thông điệp phản hồi cho người dùng sẽ được hiển thị ở vị trí dễ thấy, ở phía trên bên phải.

Về phối màu, em đã chọn một bảng màu chính với các tông màu xanh dương và lá cây và đen, tạo cảm giác dễ chịu và chuyên nghiệp. Các màu phụ được sử dụng để nhấn mạnh các yếu tố quan trọng hoặc thông báo trạng thái khác nhau như thành công, cảnh báo, lỗi và hover.

Dưới đây là một số hình ảnh minh họa cho thiết kế giao diện của các chức năng quan trọng trong ứng dụng:

\textbf{Màn hình đăng nhập}
\begin{figure}[H]
    \centering
    \includegraphics[width=1\linewidth]{Hinhve/web_login.png}
    \caption{Màn hình đăng nhập}
    \label{fig:web_login}
\end{figure}

\textbf{Màn hình đăng ký}
\begin{figure}[H]
    \centering
    \includegraphics[width=1\linewidth]{Hinhve/web_signup.png}
    \caption{Màn hình đăng ký}
    \label{fig:web_signup}
\end{figure}

\textbf{Màn hình profile}
\begin{figure}[H]
    \centering
    \includegraphics[width=1\linewidth]{Hinhve/web_profile.png}
    \caption{Màn hình profile}
    \label{fig:web_profile}
\end{figure}

\textbf{Màn hình chính}
\begin{figure}[H]
    \centering
    \includegraphics[width=1\linewidth]{Hinhve/web_home.png}
    \caption{Màn hình chính}
    \label{fig:web_home}
\end{figure}

\textbf{Màn hình gợi ý phim ở trang chủ}
\begin{figure}[H]
    \centering
    \includegraphics[width=1\linewidth]{Hinhve/web_recommend.png}
    \caption{Màn hình gợi ý phim ở trang chủ}
    \label{fig:web_recommend}
\end{figure}

\textbf{Màn hình thông tin chi tiết}
\begin{figure}[H]
    \centering
    \includegraphics[width=1\linewidth]{Hinhve/web_detail.png}
    \caption{Màn hình thông tin chi tiết 1}
    \label{fig:web_detail}
\end{figure}

\begin{figure}[H]
    \centering
    \includegraphics[width=1\linewidth]{Hinhve/web_detail1.png}
    \caption{Màn hình thông tin chi tiết 2}
    \label{fig:web_detail1}
\end{figure}

\textbf{Màn hình tổng hợp đạo diễn, diễn viên, ...}
\begin{figure}[H]
    \centering
    \includegraphics[width=1\linewidth]{Hinhve/web_cast_and_crew.png}
    \caption{Màn hình tổng hợp đạo diễn, diễn viên, ...}
    \label{fig:web_cast_and_crew}
\end{figure}

\textbf{Màn hình tổng hợp đạo diễn, diễn viên, ...}
\begin{figure}[H]
    \centering
    \includegraphics[width=1\linewidth]{Hinhve/web_cast_and_crew.png}
    \caption{Màn hình tổng hợp đạo diễn, diễn viên, ...}
    \label{fig:web_cast_and_crew}
\end{figure}

\textbf{Màn hình review phim}
\begin{figure}[H]
    \centering
    \includegraphics[width=1\linewidth]{Hinhve/web_user_review.png}
    \caption{Màn hình review phim}
    \label{fig:web_user_review}
\end{figure}

\subsection{Thiết kế cơ sở dữ liệu}
\begin{figure}[H]
\centering
\includegraphics[width=1\linewidth]{Hinhve/ERD.jpg}
\caption{Sơ đồ ERD}
\label{fig:Ket_qua}
\end{figure} 

\begin{figure}[H]
\centering
\includegraphics[width=0.3\linewidth]{Hinhve/data1.png}
\caption{Hình ảnh các table trong database}
\label{fig:Database}
\end{figure} 

\begin{figure}[H]
\centering
\includegraphics[width=0.3\linewidth]{Hinhve/data2.png}
\caption{Hình ảnh các table trong database}
\label{fig:Database}
\end{figure} 

\begin{figure}[H]
\centering
\includegraphics[width=0.3\linewidth]{Hinhve/data3.png}
\caption{Hình ảnh các table trong database}
\label{fig:Database}
\end{figure} 

\subsection{Thiết kế hệ thống crawl dữ liệu}
\textbf{IMDb:} Ở trang web này, em chia ra các thông tin để crawl như sau:

% \begin{tabular}{|c|c|} 
%  \hline
%  \textbf{Page} & \textbf{Thông tin cần lấy} \\ \hline
%  Home & Các thông tin cơ bản của một bộ phim như: Tên phim, ngày tháng năm sản xuất, mô tả, thời lượng, ... \\ \hline
%  Award & Giải thưởng của phim \\ \hline
%  Director & Đạo diễn \\ \hline
%  Cast & Các diễn viên nổi tiếng và các diễn viên khác \\ \hline
%  Storyline & Cốt truyện, thể loại \\ \hline
%  Rating & Danh sách, số lượng đánh giá của người dùng \\ \hline
%  Review & Các đánh giá phim của người dùng khác \\ \hline
% \end{tabular}

\begin{figure}[H]
\centering
\includegraphics[width=1\linewidth]{Hinhve/tb_crawl.png}
\caption{Tổng hợp các thông tin cần lấy}
\label{fig:tb_crawl}
\end{figure} 

\textbf{BoxofficeMojo:} Ở trang web này, em sẽ tìm các bộ phim có trong BoxofficeMojo bằng cách lấy khóa chính là movie name có trong IMDb, sau đó sẽ lấy các thông tin về doanh thu và update vào data.

\section{Xây dựng ứng dụng}
\subsection{Thư viện và công cụ sử dụng}
Sau đây là các công cụ, ngôn ngữ lập trình, API, thư viện, IDE, công cụ kiểm thử, v.v. mà em sử dụng để phát triển ứng dụng.

\begin{table}[H]
\centering{}
    \begin{tabular}{lll}
        \hline
        \textbf{Mục đích} & \textbf{Công cụ} & \textbf{Tài liệu tham khảo}    \\ \hline
        IDE lập trình & Visual Studio Code (VSCode) & \cite{VSCode} \\ \hline
Backend Framework & Django & \cite{Django} \\ \hline
API Framework & Django Rest Framework & \cite{DRF} \\ \hline
Frontend & HTML & \cite{HTML} \\ \hline
Styling & CSS & \cite{CSS} \\ \hline
Frontend Framework & Bootstrap & \cite{Bootstrap} \\ \hline
JavaScript Library & jQuery & \cite{jQuery} \\ \hline
Phiên bản kiểm soát & Git & \cite{Git} \\ \hline
Kho lưu trữ mã nguồn & GitHub & \cite{GitHub} \\ \hline
Công cụ crawl dữ liệu & Selenium & \cite{Selenium} \\ \hline
HTTP Library & Requests & \cite{Requests} \\ \hline
Công cụ CI/CD & GitHub Actions & \cite{GitHubActions} \\ \hline
Cơ sở dữ liệu & MySQL & \cite{MySQL} \\ \hline
        \end{tabular}
    \caption{Danh sách thư viện và công cụ sử dụng}
    \label{fig:my_label}
\end{table}

% \begin{figure}[H]
% \centering
% \includegraphics[width=1\linewidth]{Hinhve/congnghe.png}
% \caption{Danh sách thư viện và công cụ sử dụng}
% \label{fig:congnghe}
% \end{figure}

\subsection{Kết quả đạt được}
\textbf{Mô tả kết quả đạt được:}
Sau quá trình phát triển, ứng dụng Django của em đã hoàn thành với các sản phẩm được đóng gói bao gồm:

Gói Model: Chứa các mô hình dữ liệu (models) quản lý thông tin phim, người dùng, đánh giá, v.v.

Gói View: Chứa các lớp hiển thị và xử lý giao diện cho người dùng và admin.

Gói Template: Chứa các mẫu HTML dành cho giao diện người dùng và admin.

Các sản phẩm được đóng gói có ý nghĩa và vai trò cụ thể trong việc quản lý và hiển thị thông tin về phim, đạo diễn, diễn viên, đánh giá phim, và các thông tin liên quan khác. Ứng dụng này cung cấp nền tảng để người dùng có thể tìm kiếm và xem thông tin về các bộ phim, viết đánh giá, và tương tác với nội dung. Ngoài ra, ứng dụng còn có thể đưa ra những bộ phim phù hợp nhất với từng thao tác của người dùng.

\textbf{Thống kê các thông tin của ứng dụng:}
Số dòng code crawl dữ liệu: 1500 dòng
Số dòng code phía backend: 3000 dòng
Số dòng code phía fronend: 7500 dòng
Số lớp: 170 lớp
Số gói: 7 gói
Dung lượng toàn bộ mã nguồn: Khoảng 300MB khi đã có hệ thống gợi ý và dữ liệu, còn nếu mỗi phần code thì dung lượng khoảng 20MB

Ứng dụng đã hoàn thành với cấu trúc rõ ràng, mã nguồn được tổ chức tốt trong các gói và lớp. Thống kê chi tiết về số dòng code, số lớp, và dung lượng mã nguồn giúp đánh giá và theo dõi tiến trình phát triển của dự án. Ứng dụng đáp ứng các yêu cầu đặt ra và sẵn sàng để triển khai và sử dụng.

\subsection{Minh họa các chức năng chính}
\textbf{Chức năng gợi ý phim dựa vào lịch sử thao tác của người dùng}
\begin{figure}[H]
\centering
\includegraphics[width=1\linewidth]{Hinhve/ketqua1.png}
\caption{Chức năng gợi ý phim dựa vào lịch sử thao tác của người dùng}
\label{fig:ketqua1}
\end{figure} 

Dựa vào lịch sử thêm vào danh sách yêu thích của người dùng, hệ thống có thể tự học và đưa ra được kết quả mới nhất phù hợp với sở thích của người dùng.

\textbf{Chức năng gợi ý phim dựa vào nội dung phim đang xem}
\begin{figure}[H]
\centering
\includegraphics[width=1\linewidth]{Hinhve/ketqua2.png}
\caption{Chức năng gợi ý phim dựa vào nội dung phim đang xem}
\label{fig:ketqua2}
\end{figure} 

Dựa vào bộ phim mà người dùng đang tìm hiểu, hệ thống có thể đưa ra các bộ phim tương tự với bộ phim đang xem.

\textbf{Chức năng thêm phim vào danh sách yêu thích}
\begin{figure}[H]
\centering
\includegraphics[width=1\linewidth]{Hinhve/ketqua3.png}
\caption{Chức năng thêm phim vào danh sách yêu thích}
\label{fig:ketqua3}
\end{figure} 

Khi người dùng có hứng thú với một bộ phim nào đó, họ có thể thêm phim đó vào danh sách yêu thích của mình.

\textbf{Chức năng review phim}
\begin{figure}[H]
\centering
\includegraphics[width=1\linewidth]{Hinhve/ketqua4.png}
\caption{Chức năng review phim}
\label{fig:ketqua4}
\end{figure} 

Người dùng có thể đọc bình luận, đánh giá của một bộ phim và có thể tự mình đánh giá về bộ phim đó.

\textbf{Chức năng xem toàn bộ ảnh của phim}
\begin{figure}[H]
\centering
\includegraphics[width=1\linewidth]{Hinhve/ketqua5.png}
\caption{Chức năng xem toàn bộ ảnh của phim}
\label{fig:ketqua5}
\end{figure} 

Người dùng có thể xem danh sách ảnh có trong bộ phim đó, dữ liệu được tổng hợp từ các nguồn uy tín.

\section{Kiểm thử}

\FloatBarrier
\begin{figure}[ht!]
\centering
\includegraphics[width=1\linewidth]{Hinhve/test1.png}
\caption{Kiểm thử chức năng 1}
\label{fig:test1}
\end{figure} 
\FloatBarrier

\FloatBarrier
\begin{figure}[ht!]
\centering
\includegraphics[width=1\linewidth]{Hinhve/test2.png}
\caption{Kiểm thử chức năng 2}
\label{fig:test2}
\end{figure} 
\FloatBarrier

Qua những bước kiểm thử trên, có thể thấy ứng dụng đã sẵn sàng đi vào hoạt động khi toàn bộ test case đều được thông qua.

\section{Triển khai}
\textbf{Mô hình triển khai}
Ứng dụng được triển khai theo mô hình client-server với phần frontend (FE) chạy trên live server và phần backend (BE) được triển khai bằng Django trên máy tính cá nhân (local).

\textbf{Cấu hình triển khai}: Máy tính cá nhân với thông số:

Hệ điều hành: Windows/Linux/MacOS

CPU: Intel i5 hoặc tương đương

RAM: 8GB

Ổ cứng: 256GB SSD

Kết nối mạng: Internet tốc độ cao

\textbf{Các bước triển khai}
\begin{enumerate}
    \item \textbf{Triển khai frontend (FE):}

    Sử dụng Live Server trong Visual Studio Code (VS Code) để chạy ứng dụng frontend.
    
    Live Server cung cấp một máy chủ phát triển đơn giản, tự động nạp lại trang khi có thay đổi trong mã nguồn.

    \item \textbf{Triển khai backend (BE):}

    Cài đặt và cấu hình Django trên máy tính cá nhân.
    
    Cài đặt cơ sở dữ liệu MySQL.
    
    Chạy server Django bằng lệnh python manage.py runserver.

    \item \textbf{Cơ sở dữ liệu:}

    Sử dụng cơ sở dữ liệu MySQL cho môi trường phát triển.
    
    Tạo và áp dụng các migrations bằng lệnh python manage.py migrate.
\end{enumerate}

\textbf{Kết quả triển khai thử nghiệm}

Mặc dù triển khai chỉ giới hạn trong môi trường local, nhưng một số kết quả thử nghiệm đã được ghi nhận:

Khả năng chịu tải: Với môi trường local, khả năng chịu tải chưa được kiểm thử kỹ lưỡng. Tuy nhiên, với số lượng truy cập thử nghiệm nhỏ, ứng dụng hoạt động mượt mà, không gặp vấn đề về hiệu suất.

Thời gian phản hồi: Thời gian phản hồi trung bình dưới 2 giây cho các yêu cầu cơ bản (truy cập trang chủ, tìm kiếm phim, xem chi tiết phim).

Phản hồi người dùng: Người dùng thử nghiệm đánh giá cao về giao diện đơn giản, dễ sử dụng và tính năng tìm kiếm, đánh giá phim hoạt động tốt. Đặc biệt, hệ thống gợi ý cho ra kết quả rất đúng với mong muốn của người dùng.

Triển khai ứng dụng trên môi trường local đã giúp xác minh các chức năng cơ bản và thu thập phản hồi từ người dùng thử nghiệm. Các kết quả thử nghiệm cho thấy ứng dụng hoạt động ổn định, phản hồi nhanh và giao diện thân thiện với người dùng. Để kiểm thử hiệu suất và khả năng chịu tải tốt hơn, cần triển khai trên server thực tế hoặc môi trường cloud.

\end{document}
