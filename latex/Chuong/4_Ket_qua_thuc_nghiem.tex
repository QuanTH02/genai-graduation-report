\documentclass[../DoAn.tex]{subfiles}
\begin{document}

\section{Thiết kế kiến trúc}
\subsection{Lựa chọn kiến trúc phần mềm}
Trong sản phẩm này, tôi lựa chọn kiến trúc phần mềm MVT trong Django \cite{DjangoMVT}.

\textbf{Giải thích sơ bộ về MVT:}

Kiến trúc MVT của Django tương tự như kiến trúc MVC (Model-View-Controller), nhưng có sự khác biệt về cách hoạt động.

\begin{itemize}
    \item \textbf{Model (M):} Định nghĩa cấu trúc dữ liệu của ứng dụng và quản lý truy cập vào cơ sở dữ liệu.
    \item \textbf{View (V):} Xử lý logic xử lý nghiệp vụ và hiển thị dữ liệu cho người dùng.
    \item \textbf{Template (T):} Định nghĩa giao diện người dùng bằng cách sử dụng HTML cùng với các biến Python.
\end{itemize}

\textbf{Mô tả kiến trúc cụ thể cho ứng dụng:}

\begin{itemize}
    \item \textbf{Model (M):} 
    
Trong ứng dụng của em, các model như MovieInformation, Awards, Director, Cast, Genres, và các model liên quan khác sẽ đại diện cho các đối tượng dữ liệu chính về phim, đạo diễn, diễn viên, thể loại.

Mỗi model sẽ có các trường (fields) tương ứng và quan hệ (relationships) với các model khác để lưu trữ và quản lý thông tin một cách logic và cấu trúc hóa.
    
    \item \textbf{View (V):} 

Các view như FilmListView, DirectorListView, GenreListView, ReviewView, và các view khác trong danh sách của tôi sẽ đảm nhiệm xử lý các yêu cầu từ người dùng.

View sẽ truy xuất dữ liệu từ model thông qua các queryset, xử lý logic nghiệp vụ và chuẩn bị dữ liệu để truyền cho template.

    \item \textbf{Template (T):} 

Sử dụng các template HTML của Django kết hợp với template tags hiển thị dữ liệu cho người dùng.

Template sẽ nhận dữ liệu từ view và render ra các thành phần giao diện như danh sách phim, thông tin chi tiết, biểu đồ.

\end{itemize}

\subsection{Thiết kế tổng quan}
\textbf{Các gói và sự phụ thuộc giữa các gói: }

\begin{figure}[H]
    \centering
    \includegraphics[width=1\linewidth]{Hinhve/imgUML.png}
    \caption{Biểu đồ phụ thuộc gói}
    \label{fig:imgUML}
\end{figure}

\begin{enumerate}
    \item Data Access Layer (Model)

    \begin{itemize}
        \item Mục đích: Ở tầng này, chứa các định nghĩa về dữ liệu và cấu trúc dữ liệu của ứng dụng.
    
        \item Nhiệm vụ: Mô hình hóa các thực thể trong ứng dụng, bao gồm các bảng cơ sở dữ liệu và các mối quan hệ giữa chúng.
        
        \item Các thành phần: MovieInformation, Awards, Director, Cast, Genres.
        
        \item Phụ thuộc: Không phụ thuộc vào bất kỳ một gói nào khác. Đây là tầng thấp nhất trong kiến trúc, nơi dữ liệu được định nghĩa và quản lý bởi các model.
    \end{itemize}

    \item Business Logic Layer (View)

    \begin{itemize}
        \item Mục đích: Ở tầng này, xử lý logic nghiệp vụ và quản lý các yêu cầu từ người dùng.
        
        \item Nhiệm vụ: Truy xuất dữ liệu từ các model, xử lý và chuẩn bị dữ liệu để hiển thị trong template.
        
        \item Các thành phần: FilmListView, DirectorListView, GenreListView, ReviewView.
        
        \item Phụ thuộc: Tầng này phụ thuộc vào gói Models. View sẽ truy xuất dữ liệu từ model và xử lý logic nghiệp vụ ở dưới, không phụ thuộc vào gói Templates trực tiếp. Dữ liệu được chuẩn bị trong view sẽ được truyền đến template để hiển thị cho người dùng.
    \end{itemize}

    \item Presentation Layer (Template)

    \begin{itemize}
        \item Mục đích: Ở tầng này, định nghĩa giao diện người dùng.
        
        \item Nhiệm vụ: Hiển thị dữ liệu được truyền từ view dưới dạng HTML, CSS và JavaScript. Template sẽ nhận dữ liệu từ view và hiển thị nó.
        
        \item Các thành phần: Các tệp HTML template cho các trang web khác nhau của ứng dụng, ví dụ như trang đăng nhập, đăng ký, trang chủ, trang chi tiết phim.
        
        \item Phụ thuộc: Tầng này phụ thuộc vào gói Views. Templates nhận dữ liệu từ view và hiển thị nó, không phụ thuộc vào gói Models trực tiếp. Mọi dữ liệu hiển thị trên giao diện đều được truyền từ view.
    \end{itemize}

\end{enumerate}

\subsection{Thiết kế chi tiết gói}

\begin{figure}[H]
    \centering
    \includegraphics{Hinhve/layerModel.png}
    \caption{Thiết kế chi tiết gói ở tầng Model}
    \label{fig:layerModel}
\end{figure}

\textbf{Gói Model:}

Ở gói này, có các gói nhỏ hợp thành là User và Movie. 

\begin{itemize}
    \item \textbf{Gói User:} 
    Gồm 3 class l là Review, User và LikeMovie. 

    Mô tả: Class User: Đại diện cho thông tin người dùng. Class Review: Chứa thông tin về đánh giá phim của người dùng. Class LikeMovie: Lưu trữ những bộ phim mà người dùng thích.
    
    Phụ thuộc: Class Review và class LikeMovie phụ thuộc vào thông tin của User. 
    
    \item  \textbf{Gói Movie:} 
    
    Gồm 5 class là Movieinfomation, Director, Award, Cast, Writer. 

    Mô tả: Class MovieInformation: Lớp chính quản lý thông tin tổng quát về phim. Class Director: Lưu trữ thông tin chi tiết về các đạo diễn của phim. Class Award: Lưu trữ thông tin về các giải thưởng phim đã nhận được. Class Cast: Lưu trữ thông tin về các diễn viên tham gia trong phim. Class Writer: Lưu trữ thông tin về các nhà biên kịch của phim.

    Phụ thuộc: 4 class còn lại có quan hệ phụ thuộc với Movieinfomation, lấy Movieinfomation làm gốc, các thông tin còn lại đều là thông tin bổ sung.
\end{itemize}

\begin{figure}[H]
    \centering
    \includegraphics{Hinhve/layerView.png}
    \caption{Thiết kế chi tiết gói ở tầng View}
    \label{fig:layerView}
\end{figure}

\textbf{Gói View:}

Ở gói này, có các gói nhỏ hợp thành là UserView, MovieView và AdminView.

\begin{itemize}
    \item Gói UserView:

    Gồm 3 class l là ReviewView, UserView và LikeMovieView. 

    Mô tả: Class UserView: Lớp này chịu trách nhiệm hiển thị và xử lý các thông tin liên quan đến người dùng, bao gồm đăng ký, đăng nhập, và thông tin cá nhân. Class ReviewView: Lớp này xử lý và hiển thị các đánh giá phim của người dùng. Nó phụ thuộc vào thông tin người dùng từ UserView. Class LikeMovieView: Lớp này quản lý và hiển thị các bộ phim mà người dùng đã thích. Nó cũng phụ thuộc vào thông tin người dùng từ UserView.
    
    Phụ thuộc: Class ReviewView và class LikeMovieView phụ thuộc vào thông tin của UserView. 

    \item  Gói MovieView: 
    
    Gồm 5 class là MovieinfomationView, DirectorView, AwardView, CastView, WriterView. 

    Mô tả: Class MovieInformationView: Lớp chính quản lý, thao tác chính về phim. Class DirectorView: Xử lý dữ liệu về đạo diễn. Class AwardView: Xử lý dữ liệu, các thao tác về giải thưởng. Class CastView: Xử lý dữ liệu, các thao tác tính về diễn viên. Class Writer: Thao tác dữ liệu với biên kịch.

    Phụ thuộc: 4 class còn lại có quan hệ phụ thuộc với Movieinfomation, lấy Movieinfomation làm gốc, các thông tin còn lại đều là thông tin bổ sung.

    \item  Gói AdminView: 
    
    Gồm 1 class là AdminView. 

    Mô tả: Class AdminView có thể xử lý toàn bộ yêu cầu bao gồm các yêu cầu giống người dùng và có thể khác, admin quản lý toàn bộ trang web. Có thể thêm, sửa, xóa toàn bộ dữ liệu.

    Phụ thuộc: Không phụ thuộc vào class nào.
\end{itemize}

\begin{figure}[H]
    \centering
    \includegraphics[width=1\linewidth]{Hinhve/layerTemp.png}
    \caption{Thiết kế chi tiết gói ở tầng Template}
    \label{fig:layerTemp}
\end{figure}

\textbf{Gói Template:}

Gồm 2 class là UserTemplate và AdminTemplate.

Mô tả: Class UserTemplate: Lớp này chịu trách nhiệm quản lý mẫu HTML dành cho giao diện người dùng. Các mẫu này bao gồm các trang đăng nhập, đăng ký, trang cá nhân, trang đánh giá phim, và các trang hiển thị danh sách phim, diễn viên, đạo diễn. Class AdminTemplate: Lớp này chịu trách nhiệm quản lý mẫu HTML dành cho giao diện admin. Các mẫu này bao gồm các trang quản lý người dùng, quản lý phim, duyệt đánh giá và các trang khác mà admin sử dụng để quản lý ứng dụng.
    
Phụ thuộc: 2 class này không có mối quan hệ nào phụ thuộc với nhau. Chúng độc lập và phục vụ các mục đích khác nhau: một bên là giao diện người dùng, một bên là giao diện admin.

\section{Thiết kế chi tiết}
\subsection{Thiết kế giao diện}
Khi thiết kế giao diện cho ứng dụng, tôi đã chọn các thông số, tiêu chuẩn và template chuẩn để đảm bảo tính nhất quán và tiện dụng. Các màn hình mục tiêu cho ứng dụng này được thiết kế có độ phân giải 1920x1080 pixels, phù hợp với các kích thước màn hình từ 13 inch trở lên. Ứng dụng hỗ trợ hiển thị đến 16 triệu màu để đảm bảo các yếu tố đồ họa và hình ảnh hiển thị sắc nét.

Trong quá trình thiết kế, tôi đã áp dụng các quy chuẩn cụ thể để tạo ra giao diện người dùng nhất quán. Các nút bấm được thiết kế với các kích thước chuẩn chung, dễ dàng nhận biết và sử dụng. Các điều khiển như dropdowns, checkboxes, và radio buttons được thiết kế để có giao diện nhất quán, đồng nhất và rõ ràng. Các thông điệp phản hồi cho người dùng sẽ được hiển thị ở vị trí dễ thấy, ở phía trên bên phải của màn hình.

Về phối màu, tôi đã chọn một bảng màu chính với các tông màu đen và tím, tạo cảm giác dễ chịu và chuyên nghiệp. Các màu phụ được sử dụng để nhấn mạnh các yếu tố quan trọng hoặc thông báo trạng thái khác nhau như thành công, cảnh báo, lỗi và hover là đỏ, vàng và xanh lá cây.

Dưới đây là một số hình ảnh minh họa cho thiết kế giao diện của các chức năng quan trọng trong ứng dụng của tôi:

\textbf{Màn hình đăng nhập}
\begin{figure}[H]
    \centering
    \includegraphics[width=1\linewidth]{Hinhve/web_login.png}
    \caption[Màn hình \textit{đăng nhập}]{Màn hình \textit{đăng nhập}}
    \label{fig:web_login}
\end{figure}

\textbf{Màn hình đăng ký}
\begin{figure}[H]
    \centering
    \includegraphics[width=1\linewidth]{Hinhve/web_signup.png}
    \caption[Màn hình \textit{đăng ký}]{Màn hình \textit{đăng ký}}
    \label{fig:web_signup}
\end{figure}

\textbf{Màn hình profile}
\begin{figure}[H]
    \centering
    \includegraphics[width=1\linewidth]{Hinhve/web_profile.png}
    \caption[Màn hình \textit{profile}]{Màn hình \textit{profile}}
    \label{fig:web_profile}
\end{figure}

\textbf{Màn hình chính}
\begin{figure}[H]
    \centering
    \includegraphics[width=1\linewidth]{Hinhve/web_home.png}
    \caption[Màn hình \textit{chính}]{Màn hình \textit{chính}}
    \label{fig:web_home}
\end{figure}

\textbf{Màn hình gợi ý phim ở trang chủ}
\begin{figure}[H]
    \centering
    \includegraphics[width=1\linewidth]{Hinhve/web_recommend.png}
    \caption[Màn hình \textit{gợi ý phim ở trang chủ}]{Màn hình \textit{gợi ý phim ở trang chủ}}
    \label{fig:web_recommend}
\end{figure}

\textbf{Màn hình thông tin chi tiết}
\begin{figure}[H]
    \centering
    \includegraphics[width=1\linewidth]{Hinhve/web_detail.png}
    \caption[Màn hình \textit{thông tin chi tiết 1}]{Màn hình \textit{thông tin chi tiết 1}}
    \label{fig:web_detail}
\end{figure}

\begin{figure}[H]
    \centering
    \includegraphics[width=1\linewidth]{Hinhve/web_detail1.png}
    \caption[Màn hình \textit{thông tin chi tiết 2}]{Màn hình \textit{thông tin chi tiết 2}}
    \label{fig:web_detail1}
\end{figure}

\textbf{Màn hình tổng hợp đạo diễn, diễn viên, ...}
\begin{figure}[H]
    \centering
    \includegraphics[width=1\linewidth]{Hinhve/web_cast_and_crew.png}
    \caption[Màn hình \textit{tổng hợp đạo diễn, diễn viên, ...}]{Màn hình \textit{tổng hợp đạo diễn, diễn viên, ...}}
    \label{fig:web_cast_and_crew}
\end{figure}

\textbf{Màn hình review phim}
\begin{figure}[H]
    \centering
    \includegraphics[width=1\linewidth]{Hinhve/web_user_review.png}
    \caption[Màn hình \textit{review phim}]{Màn hình \textit{review phim}}
    \label{fig:web_user_review}
\end{figure}

\subsection{Thiết kế cơ sở dữ liệu}
\begin{figure}[H]
\centering
\includegraphics[width=1\linewidth]{Hinhve/ERD.jpg}
\caption{Sơ đồ ERD}
\label{fig:Ket_qua}
\end{figure} 

\begin{figure}[H]
\centering
\includegraphics[width=0.3\linewidth]{Hinhve/data1.png}
\caption{Hình ảnh các table trong database 1}
\label{fig:Database1}
\end{figure} 

\begin{figure}[H]
\centering
\includegraphics[width=0.3\linewidth]{Hinhve/data2.png}
\caption{Hình ảnh các table trong database 2}
\label{fig:Database2}
\end{figure} 

\begin{figure}[H]
\centering
\includegraphics[width=0.3\linewidth]{Hinhve/data3.png}
\caption{Hình ảnh các table trong database 3}
\label{fig:Database3}
\end{figure} 

\subsection{Thiết kế hệ thống crawl dữ liệu}
\textbf{IMDb:} Ở trang web này, tôi chia ra các thông tin để crawl như sau:

% \begin{tabular}{|c|c|} 
%  \hline
%  \textbf{Page} & \textbf{Thông tin cần lấy} \\ \hline
%  Home & Các thông tin cơ bản của một bộ phim như: Tên phim, ngày tháng năm sản xuất, mô tả, thời lượng, ... \\ \hline
%  Award & Giải thưởng của phim \\ \hline
%  Director & Đạo diễn \\ \hline
%  Cast & Các diễn viên nổi tiếng và các diễn viên khác \\ \hline
%  Storyline & Cốt truyện, thể loại \\ \hline
%  Rating & Danh sách, số lượng đánh giá của người dùng \\ \hline
%  Review & Các đánh giá phim của người dùng khác \\ \hline
% \end{tabular}

\begin{table}[H]
\centering
\caption{Tổng hợp các thông tin cần lấy}
\includegraphics[width=1\linewidth]{Hinhve/tb_crawl.png}
\label{tab:tb_crawl}
\end{table} 

\textbf{BoxofficeMojo:} Ở trang web này, tôi sẽ tìm các bộ phim có trong BoxofficeMojo bằng cách lấy khóa chính là movie name có trong IMDb, sau đó sẽ lấy các thông tin về doanh thu và update vào data.

\subsection{Tìm kiếm phim bằng ngôn ngữ tự nhiên}
Ứng dụng hỗ trợ tìm kiếm phim bằng câu tự nhiên (natural language search), cho phép người dùng nhập yêu cầu tìm kiếm bằng ngôn ngữ thông thường thay vì phải sử dụng các từ khóa cụ thể. Tính năng này kết hợp hai phương pháp tìm kiếm: Vector Search (semantic search) và AI Filter (structured search) để đảm bảo độ chính xác và toàn diện trong kết quả tìm kiếm.

\subsubsection{Tổng quan luồng xử lý}

\begin{figure}[H]
    \centering
    \includegraphics[width=1\linewidth]{Hinhve/web_search.png}
    \caption[Màn hình \textit{tìm kiếm phim}]{Màn hình \textit{tìm kiếm phim}}
    \label{fig:web_search}
\end{figure}

Khi người dùng nhập câu tìm kiếm, ứng dụng sẽ xử lý theo các bước sau:

\begin{itemize}
    \item Frontend gửi yêu cầu: Component React sẽ gửi câu tìm kiếm đến API endpoint.
    \item Backend xử lý song song: Ứng dụng thực hiện đồng thời hai phương pháp tìm kiếm:
    \begin{itemize}
        \item Vector Search: Tìm kiếm theo ngữ nghĩa (semantic) dựa trên embedding
        \item AI Filter: Tìm kiếm có cấu trúc (structured) dựa trên các thuộc tính cụ thể
    \end{itemize}
    \item Kết hợp kết quả: Ứng dụng hợp nhất và sắp xếp kết quả từ hai phương pháp theo độ ưu tiên
    \item Trả về kết quả: Gửi danh sách phim kèm thông tin phân trang về frontend
\end{itemize}

\subsubsection{Các thành phần kỹ thuật}

Frontend: Frontend nhận sự kiện click button tìm kiếm và gửi yêu cầu đến API endpoint.

Backend API Endpoint: Backend kiểm tra tính hợp lệ của query, xử lý yêu cầu tìm kiếm. Xử lý phân trang và điều phối hai phương pháp tìm kiếm song song.

Vector Search (Semantic Search):

Thực hiện tìm kiếm theo ngữ nghĩa:

\begin{itemize}
    \item Tạo embedding vector từ câu query bằng mô hình Gemini.
    \item Tải embeddings của các bộ phim đã được tính toán trước.
    \item Tính cosine similarity giữa embedding của query và embedding của mỗi phim
    \item Chọn top 200 kết quả có độ tương đồng cao nhất
\end{itemize}

Embedding của phim được tạo bởi kết hợp các thông tin: tên phim, năm sản xuất, mô tả, storyline, thể loại, đạo diễn, và diễn viên.

\textbf{AI Filter (Structured Search):}

Phân tích câu query bằng AI:

\begin{itemize}
    \item Gửi câu query đến mô hình Gemini.
    \item AI trích xuất các thông tin có cấu trúc từ câu query thành định dạng JSON, bao gồm:
    \begin{itemize}
        \item movie\_name: Tên phim

        \item year\_min/year\_max: Khoảng năm sản xuất
        \item rating\_min/rating\_max: Khoảng đánh giá
        \item genres: Danh sách thể loại
        \item directors: Danh sách đạo diễn
        \item cast: Danh sách diễn viên
        \item countries: Danh sách quốc gia
        \item languages: Danh sách ngôn ngữ
        \item keywords: Các từ khóa khác
    \end{itemize}
    \item Cuối cùng là thực hiện lọc phim dựa trên các tiêu chí được AI trích xuất
\end{itemize}

\subsubsection{Chiến lược kết hợp kết quả}

Ứng dụng hợp nhất kết quả từ hai phương pháp tìm kiếm theo ba mức độ ưu tiên:

\begin{itemize}
    \item Ưu tiên cao: Các phim xuất hiện trong cả Vector Search và AI Filter - được sắp xếp theo độ tương đồng (similarity) từ cao xuống thấp
    \item Ưu tiên trung: Các phim chỉ xuất hiện trong Vector Search - tối đa 50 kết quả
    \item Ưu tiên thấp: Các phim chỉ xuất hiện trong AI Filter - tối đa 50 kết quả
\end{itemize}

Chiến lược này đảm bảo kết quả vừa phù hợp về mặt ngữ nghĩa, vừa đáp ứng các tiêu chí cụ thể mà người dùng yêu cầu.

\subsubsection{Xử lý sau tìm kiếm}

Sau khi có danh sách phim kết quả, ứng dụng thực hiện các bước xử lý sau:

\begin{itemize}
    \item Serialize dữ liệu: Chuyển đổi đối tượng phim sang định dạng JSON để trả về cho frontend
    \item Lưu lịch sử tìm kiếm: Nếu người dùng đã đăng nhập, thì hệ thống sẽ lưu lại lịch sử tìm kiếm với định dạng JSON để phục vụ cho ứng dụng gợi ý phim trong tương lai
\end{itemize}

\subsubsection{Kết quả trả về}

API trả về response với cấu trúc:

\begin{itemize}
    \item data: Danh sách phim đã được sắp xếp và lọc
    \item pagination: Thông tin phân trang (số trang hiện tại, tổng số trang, tổng số kết quả)
    \item filter\_data: Thông tin các bộ lọc mà AI đã phân tích từ câu query, giúp người dùng hiểu rõ h
\end{itemize}

\subsubsection{Điểm mạnh của ứng dụng}

Tính năng tìm kiếm bằng ngôn ngữ người dùng có các ưu điểm sau:

\begin{itemize}
    \item Kết hợp hai phương pháp: Sự kết hợp giữa semantic search và structured search giúp cải thiện đáng kể độ chính xác của kết quả tìm kiếm
    \item Hỗ trợ đa ngôn ngữ: Thông qua mô hình Gemini, ứng dụng có thể xử lý câu tìm kiếm bằng nhiều ngôn ngữ khác nhau
    \item Lưu lịch sử: Ứng dụng tự động lưu lịch sử tìm kiếm của người dùng đã đăng nhập, hỗ trợ cho các tính năng gợi ý và phân tích hành vi
    \item Tối ưu hiệu suất: Embeddings được tính toán trước và lưu trong file pickle, giúp giảm thời gian xử lý khi tìm kiếm
\end{itemize}


\subsection{Gợi ý phim bằng genAI dựa vào lịch sử hoạt động của người dùng}
Ứng dụng sử dụng GenAI (Gemini hoặc OpenAI) để phân tích lịch sử hoạt động của người dùng và đưa ra gợi ý phim cá nhân hóa theo thời gian thực. Khác với các phương pháp gợi ý truyền thống yêu cầu training model, hệ thống này hoạt động real-time dựa trên dữ liệu lịch sử của từng người dùng, không cần retrain hay cập nhật model.

\subsubsection{Tổng quan luồng xử lý}

\begin{figure}[H]
    \centering
    \includegraphics[width=1\linewidth]{Hinhve/web_recommend.png}
    \caption[Màn hình \textit{gợi ý phim bằng genAI}]{Màn hình \textit{gợi ý phim bằng genAI}}
    \label{fig:web_recommend}
\end{figure}

Khi người dùng yêu cầu gợi ý phim, ứng dụng sẽ xử lý theo các bước sau:

\begin{itemize}
    \item Thu thập lịch sử người dùng: Ứng dụng lấy lịch sử tìm kiếm và lịch sử đánh giá của người dùng
    \item Xây dựng prompt cho GenAI: Ứng dụng tạo prompt chứa thông tin lịch sử và yêu cầu AI phân tích sở thích
    \item Gọi GenAI API: Ứng dụng gửi prompt đến GenAI (ưu tiên Gemini, fallback OpenAI) để nhận danh sách phim được gợi ý
    \item Tìm phim trong database: Ứng dụng tìm các phim được AI gợi ý trong database
    \item Trả về kết quả: Gửi danh sách phim đã được gợi ý về frontend
\end{itemize}

\subsubsection{Các thành phần kỹ thuật}

\textbf{Frontend:} Frontend gửi yêu cầu gợi ý phim đến API endpoint và hiển thị kết quả trả về.

\textbf{Backend API Endpoint:} Backend xử lý yêu cầu, thu thập lịch sử người dùng và gọi GenAI để tạo gợi ý.

\textbf{Thu thập lịch sử hoạt động:}

Ứng dụng thu thập ba loại dữ liệu lịch sử:

\begin{itemize}
    \item Lịch sử tìm kiếm: Tự động lưu lại các câu tìm kiếm của người dùng, bao gồm cả tìm kiếm thường và tìm kiếm ngôn ngữ tự nhiên. Ứng dụng lấy tối đa 20 query gần nhất để phân tích
    \item Lịch sử đánh giá: Lưu trữ các review phim của người dùng, bao gồm điểm đánh giá (1-5 sao), tiêu đề và nội dung review. Ứng dụng lấy tối đa 20 review gần nhất, ưu tiên các review có rating cao (4-5 sao)
    \item Theo dõi hoạt động: Ghi nhận các hành động của người dùng như xem chi tiết phim, click vào card phim, xem trailer, hoặc click vào kết quả tìm kiếm. Mỗi loại hoạt động có mức độ quan trọng khác nhau
\end{itemize}

\textbf{Xây dựng prompt cho GenAI:}

Ứng dụng tạo prompt chứa:

\begin{itemize}
    \item Lịch sử tìm kiếm: Liệt kê các câu tìm kiếm gần nhất của người dùng
    \item Lịch sử đánh giá: Liệt kê các phim đã được người dùng đánh giá kèm điểm số và tiêu đề review
    \item Yêu cầu phân tích: Genres người dùng tìm kiếm, themes và keywords, phim được đánh giá tích cực (4-5 sao), phim tương tự những gì đã review, xu hướng trong pattern tìm kiếm
    \item Yêu cầu trả về: Danh sách tên các phim được gợi ý dưới dạng JSON array
\end{itemize}

\textbf{Gọi GenAI API:}

Ứng dụng ưu tiên sử dụng Gemini API để phân tích lịch sử và đưa ra gợi ý. Nếu Gemini không khả dụng, ứng dụng sẽ fallback sang OpenAI API để đảm bảo tính khả dụng của hệ thống.

Xử lý response từ GenAI:
\begin{itemize}
    \item Parse JSON từ response (loại bỏ markdown code blocks nếu có)
    \item Validate format (phải là array chứa tên phim)
    \item Giới hạn số lượng phim theo yêu cầu
\end{itemize}

\textbf{Tìm phim trong database:}

Với mỗi tên phim được AI gợi ý, ứng dụng tìm phim tương ứng trong database:
\begin{itemize}
    \item Nếu phim không tìm thấy: Bỏ qua phim đó
    \item Chỉ trả về danh sách phim hợp lệ có trong database
\end{itemize}

\subsubsection{Kết quả trả về}

API trả về response với cấu trúc:
\begin{itemize}
    \item message: Thông báo trạng thái
    \item data: Danh sách phim đã được gợi ý dưới dạng JSON array
\end{itemize}


\section{Xây dựng ứng dụng}
\subsection{Thư viện và công cụ sử dụng}
Sau đây là các công cụ, ngôn ngữ lập trình, API, thư viện, IDE, công cụ kiểm thử mà tôi sử dụng để phát triển ứng dụng này.

\begin{table}[H]
\centering{}
\caption{Danh sách thư viện và công cụ sử dụng}
    \begin{tabular}{lll}
        \hline
        \textbf{Mục đích} & \textbf{Công cụ} & \textbf{Tài liệu tham khảo}    \\ \hline
        IDE lập trình & Visual Studio Code (VSCode) & \cite{VSCode} \\ \hline
API Framework & Django Rest Framework & \cite{DRF} \\ \hline
Frontend & HTML & \cite{HTML} \\ \hline
Style & Tailwind CSS & \cite{TailwindCSS} \\ \hline
Frontend Framework & React & \cite{React} \\ \hline
Backend Framework & Django & \cite{Django} \\ \hline
Database & MySQL & \cite{MySQL} \\ \hline
Công cụ crawl dữ liệu & Selenium & \cite{Selenium} \\ \hline
Thư viện HTTP & Requests & \cite{Requests} \\ \hline
Công cụ CI/CD & GitHub Actions & \cite{GitHubActions} \\ \hline
        \end{tabular}
    \label{fig:my_label}
\end{table}

% \begin{figure}[H]
% \centering
% \includegraphics[width=1\linewidth]{Hinhve/congnghe.png}
% \caption{Danh sách thư viện và công cụ sử dụng}
% \label{fig:congnghe}
% \end{figure}

\subsection{Kết quả đạt được}
\textbf{Mô tả kết quả đạt được:}
Sau quá trình phát triển, ứng dụng Django của tôi đã hoàn thành với các sản phẩm được đóng gói bao gồm:

Gói Model: Chứa các mô hình dữ liệu quản lý thông tin phim, người dùng, đánh giá.

Gói View: Chứa các class hiển thị và xử lý giao diện cho người dùng và admin.

Gói Template: Chứa các template HTML dành cho giao diện người dùng và admin.

Các sản phẩm được đóng gói có ý nghĩa và vai trò cụ thể trong việc quản lý và hiển thị thông tin về phim, đạo diễn, diễn viên, đánh giá phim. Ứng dụng này cung cấp nền tảng để người dùng có thể tìm kiếm và xem thông tin về các bộ phim, viết đánh giá, tương tác với nội dung. Ngoài ra, ứng dụng còn có thể đưa ra những bộ phim phù hợp nhất với từng thao tác của người dùng theo gợi ý của ứng dụng.

\textbf{Thống kê các thông tin của ứng dụng:}

Số dòng code crawl dữ liệu: 2000 dòng

Số dòng code phía backend: 4000 dòng

Số dòng code phía fronend: 8000 dòng

Số lớp: 180 lớp

Số gói: 8 gói

Dung lượng toàn bộ mã nguồn: Khoảng 310MB khi đã có hệ thống gợi ý và dữ liệu, còn nếu mỗi phần code thì dung lượng khoảng 25MB

Ứng dụng đã hoàn thành với tổ chức cấu trúc rõ ràng, mã nguồn được tổ chức tốt trong các gói và lớp. Thống kê chi tiết về số dòng code, số lớp và dung lượng mã nguồn giúp đánh giá và theo dõi tiến trình phát triển của dự án. Ứng dụng đáp ứng các yêu cầu đặt ra và sẵn sàng để triển khai và sử dụng trong thực tế.

\textbf{Crawl dữ liệu từ trang web:}

Nguồn dữ liệu: IMDb, BoxOfficeMojo

Số bộ phim đã crawl: 5000 bộ phim

Số ảnh đã crawl: 250000 ảnh

Số video đã crawl: 5000 video

\textbf{Tìm kiếm phim theo ngôn ngữ người dùng:}

Ở chức năng này, người dùng có thể tìm kiếm phim theo ngôn ngữ của mình. Ứng dụng sẽ nhận đầu vào, phân tích, embed text vào model để tìm kiếm phim theo ngôn ngữ của người dùng và đưa ra kết quả phù hợp.

\begin{figure}[H]
\centering
\includegraphics[width=1\linewidth]{Hinhve/search_phim.png}
\caption{Chức năng tìm kiếm phim theo ngôn ngữ người dùng}
\label{fig:search_phim}
\end{figure} 

Dựa vào hình ảnh kết quả trên, có thể thấy hệ thống đã đưa ra kết quả phù hợp với yêu cầu của người dùng. Ở trên là các phim marvel hệ thống đã đưa ra kết quả.

\textbf{Gợi ý phim theo lịch sử hoạt động của người dùng:}

Ở chức năng này, hệ thống sẽ gợi ý phim dựa vào lịch sử hoạt động của người dùng. Hệ thống sẽ theo dõi lịch sử hoạt động của người dùng như review phim, tìm kiếm phim, xem chi tiết phim, ... và đưa ra gợi ý các bộ phim tương tự với lịch sử hoạt động của người dùng.

\begin{figure}[H]
\centering
\includegraphics[width=1\linewidth]{Hinhve/review_phim.png}
\caption{Review phim}
\label{fig:review_phim}
\end{figure} 

\begin{figure}[H]
\centering
\includegraphics[width=1\linewidth]{Hinhve/goi_y_theo_review.png}
\caption{Gợi ý phim theo lịch sử review}
\label{fig:goi_y_theo_review}
\end{figure} 

Dựa vào lịch sử review của người dùng, hệ thống có thể tự học và đưa ra được kết quả mới nhất phù hợp với sở thích của người dùng. Ở phần trên, review phim người dùng bảo thích phim Iron Man, hệ thống đã đưa ra gợi ý phim tương tự là Iron man 1, 2, 3 và các bộ phim Marvel liên quan.

\begin{figure}[H]
\centering
\includegraphics[width=1\linewidth]{Hinhve/search_phim.png}
\caption{Tìm kiếm phim}
\label{fig:search_phim2}
\end{figure} 

\begin{figure}[H]
\centering
\includegraphics[width=1\linewidth]{Hinhve/goi_y_theo_search.png}
\caption{Gợi ý phim theo lịch sử tìm kiếm}
\label{fig:goi_y_theo_search}
\end{figure} 

Dựa vào lịch sử tìm kiếm của người dùng, hệ thống lưu lại lịch sử và đưa ra được kết quả mới nhất phù hợp với yêu cầu của người dùng. Ở phần trên, người dùng tìm kiếm phim Marvel, thì sau khi ra màn Home, hệ thống đã đưa ra gợi ý các phim Marvel liên quan.

\subsection{Minh họa các chức năng chính}
\textbf{Chức năng gợi ý phim dựa vào lịch sử thao tác của người dùng}
\begin{figure}[H]
\centering
\includegraphics[width=1\linewidth]{Hinhve/ketqua1.png}
\caption{Chức năng gợi ý phim dựa vào lịch sử thao tác của người dùng}
\label{fig:ketqua1}
\end{figure} 

Dựa vào lịch sử thêm vào danh sách yêu thích của người dùng, ứng dụng có thể tự học và đưa ra được kết quả mới nhất phù hợp với sở thích của người dùng.

\textbf{Chức năng gợi ý phim dựa vào nội dung phim đang xem}
\begin{figure}[H]
\centering
\includegraphics[width=1\linewidth]{Hinhve/ketqua2.png}
\caption{Chức năng gợi ý phim dựa vào nội dung phim đang xem}
\label{fig:ketqua2}
\end{figure} 

Dựa vào bộ phim mà người dùng đang tìm hiểu, ứng dụng có thể đưa ra các bộ phim tương tự với bộ phim đang xem.

\textbf{Chức năng thêm phim vào danh sách yêu thích}
\begin{figure}[H]
\centering
\includegraphics[width=1\linewidth]{Hinhve/ketqua3.png}
\caption{Chức năng thêm phim vào danh sách yêu thích}
\label{fig:ketqua3}
\end{figure} 

Khi người dùng có hứng thú với một bộ phim nào đó, họ có thể nhấn nút thêm vào danh sách yêu thích để thêm phim đó vào danh sách yêu thích của mình.

\textbf{Chức năng review phim}
\begin{figure}[H]
\centering
\includegraphics[width=1\linewidth]{Hinhve/ketqua4.png}
\caption{Chức năng review phim}
\label{fig:ketqua4}
\end{figure} 

Người dùng vào chi tiết phim để đọc bình luận, đánh giá của một bộ phim và có thể tự mình đánh giá về bộ phim đó.

\textbf{Chức năng xem toàn bộ ảnh của phim}
\begin{figure}[H]
\centering
\includegraphics[width=1\linewidth]{Hinhve/ketqua5.png}
\caption{Chức năng xem toàn bộ ảnh của phim}
\label{fig:ketqua5}
\end{figure} 

Người dùng vào chi tiết phim để xem danh sách ảnh có trong bộ phim đó, dữ liệu được tổng hợp từ các nguồn uy tín.

\section{Kiểm thử}

\FloatBarrier
\begin{table}[ht!]
\centering
\caption{Kiểm thử chức năng}
\includegraphics[width=1\linewidth]{Hinhve/test1.png}
\label{tab:test1}
\end{table} 
\FloatBarrier

\FloatBarrier
\begin{table}[ht!]
\centering
\caption{Kiểm thử chức năng}
\includegraphics[width=1\linewidth]{Hinhve/test2.png}
\label{tab:test2}
\end{table} 
\FloatBarrier

\begin{table}[H]
\centering
\caption{Bảng danh sách kiểm thử hiệu năng}
\label{tab:performance-test-list}
\begin{tabular}{|p{0.4\textwidth}|p{0.35\textwidth}|p{0.15\textwidth}|p{0.1\textwidth}|}
\hline
\textbf{Tên kiểm thử} & \textbf{Endpoint/Method} & \textbf{Yêu cầu} & \textbf{Mã} \\
\hline
GET Danh sách phim & GET /api/v1/movies/ & < 3s & TC01 \\
\hline
GET Chi tiết phim & POST /api/v1/movies/1/ & < 3s & TC02 \\
\hline
GET Lọc phim & GET /api/v1/movies/filter-options/ & < 3s & TC03 \\
\hline
GET Metadata phim & GET /api/v1/movies/1/cast/ & < 3s & TC04 \\
\hline
POST Đăng ký & POST /api/v1/users/register/ & < 3s & TC05 \\
\hline
POST Đăng nhập & POST /api/v1/users/login/ & < 3s & TC06 \\
\hline
Tìm kiếm AI (Mock) & POST /api/v1/movies/search/natural/ & < 10s & TC07 \\
\hline
Gợi ý AI (Mock) & GET /api/v1/movies/recommend/realtime/perf\_test\_user/ & < 10s & TC08 \\
\hline
Định dạng response nhất quán & GET /api/v1/movies/ & Có & TC09 \\
\hline
Thông báo lỗi rõ ràng & POST /api/v1/movies/999999/ & Có & TC10 \\
\hline
Validation đầu vào & POST /api/v1/users/login/ & Có & TC11 \\
\hline
Documentation trong code & Code structure & Có & TC12 \\
\hline
Tổ chức code theo modules & Code organization & Có & TC13 \\
\hline
Nhất quán RESTful API & Multiple endpoints & Có & TC14 \\
\hline
Kết nối database & Database connection & < 1s & TC15 \\
\hline
Hiệu năng query cơ bản & SELECT query & < 1s & TC16 \\
\hline
Tính toàn vẹn dữ liệu & Data integrity & Có & TC17 \\
\hline
Xử lý transaction & Transaction & Có & TC18 \\
\hline
\end{tabular}
\end{table}

\begin{table}[H]
\centering
\caption{Bảng kết quả kiểm thử hiệu năng}
\label{tab:performance-test-results}
\begin{tabular}{|p{0.12\textwidth}|p{0.12\textwidth}|p{0.18\textwidth}|p{0.12\textwidth}|p{0.12\textwidth}|p{0.24\textwidth}|}
\hline
\textbf{Mã} & \textbf{Yêu cầu} & \textbf{Thời gian thực tế (s)} & \textbf{Status Code} & \textbf{Kết quả} \\
\hline
TC01 & < 3s & 0.009 & 200 & Đạt \\
\hline
TC02 & < 3s & 0.004 & 200 & Đạt \\
\hline
TC03 & < 3s & 0.099 & 200 & Đạt \\
\hline
TC04 & < 3s & 0.007 & 200 & Đạt \\
\hline
TC05 & < 3s & 0.001 & 400 & Đạt \\
\hline
TC06 & < 3s & 0.184 & 200 & Đạt \\
\hline
TC07 & < 10s & 4.6 & 200 & Đạt \\
\hline
TC08 & < 10s & 2.9 & 200 & Đạt \\
\hline
TC09 & Có & 0.000 & 200 & Đạt \\
\hline
TC10 & Có & 0.000 & 404 & Đạt \\
\hline
TC11 & Có & 0.000 & 400 & Đạt \\
\hline
TC12 & Có & 0.000 & 200 & Đạt \\
\hline
TC13 & Có & 0.000 & 200 & Đạt \\
\hline
TC14 & Có & 0.000 & 200 & Đạt \\
\hline
TC15 & < 1s & 0.000 & 200 & Đạt \\
\hline
TC16 & < 1s & 0.005 & 200 & Đạt \\
\hline
TC17 & Có & 0.000 & 200 & Đạt \\
\hline
TC18 & Có & 0.000 & 200 & Đạt \\
\hline
\end{tabular}
\end{table}
    

Qua những bước kiểm thử trên (\autoref{tab:test1}, \autoref{tab:test2}, \autoref{tab:performance-test-list}, \autoref{tab:performance-test-results}), có thể thấy ứng dụng đã được kiểm thử thành công và đã sẵn sàng đi vào hoạt động khi toàn bộ test case đều được thông qua và hoạt động tốt.

\section{Triển khai}
\textbf{Mô hình triển khai}
Ứng dụng được triển khai theo mô hình client-server với phần frontend chạy trên live server và phần backend được triển khai bằng Django trên máy tính cá nhân.

\textbf{Cấu hình triển khai}: Máy tính cá nhân với thông số:

Hệ điều hành: Windows, Linux, MacOS

CPU: Intel i3 hoặc tương đương

RAM: 4GB

Ổ cứng: 128GB SSD

Kết nối mạng: Internet tốc độ trung bình

\textbf{Triển khai}

Clone repository từ GitHub về máy tính cá nhân.

Cài đặt docker vào máy.

Cd vào thư mục project và chạy lệnh docker-compose up -d để chạy ứng dụng. Lúc này thì toàn bộ ứng dụng đã được chạy trên máy tính cá nhân. Vào cổng localhost:3000 để test ứng dụng.

\textbf{Kết quả triển khai thử nghiệm}

Mặc dù triển khai chỉ giới hạn trong môi trường local, nhưng một số kết quả thử nghiệm đã được thử nghiệm:

Khả năng chịu tải: Với môi trường local, khả năng chịu tải chưa được thử nghiệm kỹ lưỡng. Tuy nhiên, với số lượng truy cập thử nghiệm nhỏ, ứng dụng hoạt động mượt mà, không gặp vấn đề về hiệu suất.

Thời gian phản hồi: Thời gian phản hồi trung bình dưới 2 giây cho các yêu cầu cơ bản, còn gợi ý phim thì trung bình dưới 5s, tìm kiếm phim thì dưới 10s. 

Phản hồi người dùng: Người dùng thử nghiệm đánh giá cao về giao diện đơn giản, dễ sử dụng và tính năng tìm kiếm, đánh giá ứng dụng hoạt động tốt. Đặc biệt, chức năng tìm kiếm phim và gợi ý cho ra kết quả rất đúng với mong muốn của người dùng.

Triển khai ứng dụng trên môi trường local đã giúp xác minh các chức năng cơ bản và thu thập được phản hồi khá tốt từ người dùng thử nghiệm. Các kết quả thử nghiệm cho thấy ứng dụng hoạt động ổn định, phản hồi nhanh và giao diện khá thân thiện với người dùng. Để kiểm thử hiệu suất và khả năng chịu tải tốt hơn, cần triển khai trên server thực tế.

\end{document}
