\documentclass[../DoAn.tex]{subfiles}
\begin{document}

Chương này tập trung trình bày các vấn đề đã phát sinh trong quá trình thực hiện và các giải pháp cụ thể đã được áp dụng để giải quyết những vấn đề đó. Bên cạnh đó, chương cũng đề cập đến các đóng góp nổi bật được đánh giá cao trong suốt quá trình thực hiện đồ án tốt nghiệp.

\section{Thiếu khá nhiều dữ liệu phim ảnh}
Vấn đề này xuất hiện khi cần một khối lượng dữ liệu lớn về các bộ phim để phục vụ việc xây dựng ứng dụng gợi ý. Để khắc phục vấn đề này, tôi đã thực hiện tìm kiếm, thu thập và lựa chọn các trang web đáng tin cậy để thu thập dữ liệu. Quá trình này bao gồm việc phân tích cú pháp của các trang web, xác định phương thức truy cập, nghiên cứu cấu trúc và thực hiện thu thập dữ liệu một cách tự động.

\section{Thiết kế database khá khó khăn với quá nhiều trường}

Trong quá trình thiết kế cơ sở dữ liệu, tôi gặp phải thách thức khi phải xử lý một khối lượng lớn các trường thông tin liên quan đến phim và người dùng. Để khắc phục vấn đề này, tôi đã vận dụng các nguyên tắc chuẩn hóa cơ sở dữ liệu và phân tách các trường thông tin có liên quan thành các bảng dữ liệu con. Tiếp theo, tôi xác định các mối quan hệ giữa các bảng và liên kết chúng lại với nhau nhằm tạo ra một cấu trúc cơ sở dữ liệu hợp lý và đạt hiệu quả cao.

\section{Trang web có dữ liệu update liên tục theo ngày khiến link ảnh và trailer bị hỏng}

Một trong các trang web được sử dụng để thu thập dữ liệu là IMDb, trang web này chứa rất nhiều trường thông tin, tuy nhiên việc khó khăn nhất trong quá trình thu thập dữ liệu đối với tôi chính là việc lấy các liên kết ảnh và liên kết trailer. Trên trang web này, cứ mỗi ngày các liên kết ảnh và trailer sẽ được làm mới một lần, do đó các liên kết cũ sẽ trở nên không hợp lệ và không thể truy cập được vào ngày hôm sau. Tôi đã đề xuất hai phương án để giải quyết vấn đề này: Tải toàn bộ ảnh và trailer về máy hoặc tìm cách để hệ thống tự động cập nhật dữ liệu.

Về phương án tải xuống toàn bộ hình ảnh và video trailer, tôi đã từ bỏ ngay khi quyết định số lượng phim cần thiết cho bản demo. Nếu có 5000 bộ phim thì sẽ có hơn 250000 hình ảnh và 5000 video demo, việc tải xuống sẽ khiến máy tính trở nên cồng kềnh và không hiệu quả. Vì vậy, tôi đã tìm hiểu và phát hiện ra một giải pháp để hệ thống có thể tự động cập nhật dữ liệu, đó là việc sử dụng GitHub Actions.

\textbf{Giới thiệu Github Action:}

GitHub Actions là một dịch vụ được GitHub cung cấp nhằm tự động hóa các quy trình phát triển phần mềm. Thông qua việc sử dụng các action, tôi có thể thiết lập các workflow tự động, bao gồm việc kiểm tra mã nguồn, triển khai ứng dụng, và tự động hóa các tác vụ như cập nhật dữ liệu trong hệ thống.

\textbf{Triển khai:}

Ở bước đầu tiên, tôi xây dựng một file crawlJson.py nhằm thu thập toàn bộ dữ liệu mới từ website IMDb và ghi vào file update.sql, một khi đã có file update.sql thì chỉ việc import vào môi trường local là có thể có dữ liệu mới ngay.

Nhằm cho phép hệ thống tự động cập nhật dữ liệu, quy trình thực hiện bao gồm các bước sau:

\begin{enumerate}
    \item Tạo PAT (Personal Access Token) Github

    Personal Access Token (PAT) trên GitHub được sử dụng để cấp quyền truy cập cho các ứng dụng hoặc dịch vụ bên thứ ba vào tài khoản GitHub mà không cần sử dụng mật khẩu. Trong hệ thống này, dịch vụ bên thứ ba chính là bot tự động.

    \begin{figure}[H]
        \centering
        \includegraphics[width=1\linewidth]{Hinhve/gh_at_1.png}
        \caption{Tạo PAT Github}
        \label{fig:gh_at_1}
    \end{figure}
    
    \item Tạo Secrets Repository

    Mục đích của việc tạo Secrets trên GitHub là nhằm bảo vệ và quản lý các thông tin nhạy cảm như mã token, mật khẩu, khóa API cùng các thông tin quan trọng khác trong quá trình phát triển và triển khai hệ thống. Việc sử dụng Secrets góp phần đảm bảo rằng các thông tin này không bị lộ ra ngoài công cộng, chỉ có thể được truy cập và sử dụng bởi các công cụ tự động hóa như GitHub Actions hoặc các ứng dụng khác thông qua API được cấp phép.

    \begin{figure}[H]
        \centering
        \includegraphics[width=1\linewidth]{Hinhve/gh_at_2.png}
        \caption{Tạo Secrets Repository}
        \label{fig:gh_at_2}
    \end{figure}
    
    \item Viết file update.yml

    Bước cuối cùng, tôi tạo file update.yml để GitHub Actions có thể thực thi tự động hàng ngày
\end{enumerate}

\textbf{Kết quả:}

Vào lúc 00h00 mỗi ngày, hệ thống sẽ tự động thực hiện crawl dữ liệu mới, lưu vào file update.sql và đẩy code lên GitHub trong khoảng thời gian 1 giờ.

\begin{figure}[H]
    \centering
    \includegraphics[width=1\linewidth]{Hinhve/gh_at_3.png}
    \caption{Hệ thống tự động crawl dữ liệu}
    \label{fig:gh_at_3}
\end{figure}

Sau khi hệ thống tạo ra file update.sql, chỉ cần pull code về máy chủ và thực thi file Crawl/autoUpdate/update.py thì sẽ có dữ liệu mới ngay lập tức.

\begin{figure}[H]
    \centering
    \includegraphics[width=1\linewidth]{Hinhve/gh_at_4.png}
    \caption{Cập nhật dữ liệu mới thành công}
    \label{fig:gh_at_4}
\end{figure}

\section{Lựa chọn thuật toán gợi ý phù hợp}

Việc lựa chọn thuật toán gợi ý phù hợp là một thách thức quan trọng trong quá trình phát triển ứng dụng gợi ý phim. Tôi đã tham khảo các bài báo và tài liệu trên mạng, thực hiện thử nghiệm và đánh giá các thuật toán khác nhau để tìm ra thuật toán phù hợp nhất với yêu cầu của dự án. Cụ thể chính là thuật toán Content-based.

\textbf{Giới thiệu thuật toán Content-based:}

Trong hệ thống content-based này, dựa trên nội dung của mỗi movie, tôi xây dựng một bộ hộ sơ cho mỗi movie. Hồ sơ này được biểu diễn dưới dạng toán học là một feature vector. Features của một bộ phim trong hệ thống gợi ý tôi xây dựng này là:

\begin{itemize}
    \item Movie name: Tên của bộ phim đó
    \item Genres: Thể loại của bộ phim đó
    \item Describe: Mô tả sơ lược về nội dung của bộ phim đó
    \item Cast: Diễn viên của bộ phim đó
    \item Director: Đạo diễn của bộ phim đó
\end{itemize}

\textbf{Phân tích, giải pháp:}

Content-based là thuật toán mà khi chọn một item, thì sẽ đưa ra các item tương tự có những đặc điểm cụ thể giống với item đã chọn. Vì thế trong bài toán này tôi đã chọn cách tính độ tương đồng giữa các bộ phim dựa trên các feature ở trên.

\[
\text{similarity}(A, B) = \cos(\theta) = \frac{A \cdot B}{\|A\| \|B\|}
\]

Trong đó:
\begin{itemize}
  \item \(A\) và \(B\) là hai vector đại diện cho hai phim hoặc hai người dùng.
  \item \(A \cdot B\) là tích vô hướng của hai vector \(A\) và \(B\).
  \item \(\|A\|\) và \(\|B\|\) lần lượt là độ lớn của vector \(A\) và vector \(B\).
\end{itemize}

\textbf{Kiểm tra tính chính xác:}

Sau khi tính toán được độ tương đồng dựa vào công thức similarity trên, tôi kiểm tra tính chính xác của thuật toán bằng cách: Tính toán độ tương đồng riêng lẻ của các đặc trưng như movie name, genres, cast, director, describe. Sau đó kiểm tra kết quả cuối cùng với các kết quả riêng biệt xem có chính xác không.

\begin{figure}[H]
    \centering
    \includegraphics[width=1\linewidth]{Hinhve/rcm_ct_kq_1.png}
    \caption{Kiểm tra kết quả 1}
    \label{fig:rcm_ct_kq_1}
\end{figure}

Từ hình ảnh \ref{fig:rcm_ct_kq_1}, có thể nhận thấy rằng bộ phim được hệ thống đề xuất có mức độ tương đồng cao nhất với bộ phim Guardians of the Galaxy chính là bộ phim Guardians of the Galaxy Vol. 3. Hai bộ phim này có sự tương đồng về tên phim, mô tả, diễn viên và đạo diễn.

\begin{figure}[H]
    \centering
    \includegraphics[width=1\linewidth]{Hinhve/rcm_ct_kq_2.png}
    \caption{Kiểm tra kết quả 2}
    \label{fig:rcm_ct_kq_2}
\end{figure}

Từ hình ảnh \ref{fig:rcm_ct_kq_2}, có thể quan sát thấy bộ phim được hệ thống đề xuất có mức độ tương đồng cao nhất với bộ phim The Hunger Games: Catching Fire là bộ phim The Hunger Games. Hai bộ phim này thể hiện sự tương đồng đáng kể về tên phim, mô tả, diễn viên và đạo diễn.


\begin{figure}[H]
    \centering
    \includegraphics[width=1\linewidth]{Hinhve/rcm_ct_kq_3.png}
    \caption{Kiểm tra kết quả 3}
    \label{fig:rcm_ct_kq_3}
\end{figure}

Quan sát hình ảnh \ref{fig:rcm_ct_kq_3}, có thể kết luận rằng bộ phim được hệ thống gợi ý có độ tương đồng lớn nhất với bộ phim Meg 2: The Trench chính là bộ phim The Meg. Hai bộ phim này có nhiều điểm tương đồng về tên gọi, nội dung mô tả, dàn diễn viên và đạo diễn.

Dựa trên các kết quả đã được trình bày ở trên kết hợp với việc kiểm tra bằng phương pháp trực quan, có thể kết luận rằng hệ thống gợi ý phim dựa trên độ tương đồng đã được thực hiện với mức độ chính xác cao.

\end{document}