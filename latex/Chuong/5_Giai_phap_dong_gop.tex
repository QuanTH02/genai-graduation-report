\documentclass[../DoAn.tex]{subfiles}
\begin{document}

Chương này tập trung phân tích các thách thức nảy sinh trong giai đoạn triển khai và những biện pháp kỹ thuật tương ứng đã được áp dụng. Đồng thời, chương cũng tổng kết các kết quả đạt được và những đóng góp then chốt trong suốt quá trình hoàn thiện đồ án tốt nghiệp.

\section{Sự thiếu hụt trầm trọng về dữ liệu điện ảnh} Khó khăn này phát sinh khi dự án yêu cầu một kho dữ liệu khổng lồ để vận hành thuật toán gợi ý một cách hiệu quả. Để giải quyết, tôi đã tiến hành khảo sát và lựa chọn các nền tảng trực tuyến uy tín làm nguồn tài nguyên chính. Quy trình xử lý bao gồm việc bóc tách cú pháp web, thiết lập giao thức truy cập, phân tích sơ đồ cấu trúc dữ liệu và triển khai các công cụ thu thập tự động để tối ưu hóa thời gian.

\section{Thách thức trong việc thiết kế cơ sở dữ liệu với hệ thống thuộc tính phức tạp}

Việc quản lý một lượng lớn các trường thông tin liên quan đến tác phẩm và người dùng đã tạo ra áp lực lớn trong khâu thiết kế sơ đồ dữ liệu. Nhằm khắc phục tình trạng này, tôi đã áp dụng các quy chuẩn hóa dữ liệu và chia tách các thuộc tính thành các bảng phụ chuyên biệt. Sau đó, bằng cách xác lập các mối quan hệ và liên kết logic giữa các bảng, tôi đã xây dựng thành công một cấu trúc cơ sở dữ liệu khoa học, đảm bảo tính toàn vẹn và hiệu suất truy xuất cao.

\section{Trang web có dữ liệu update liên tục theo ngày khiến link ảnh và trailer bị hỏng}

Một trong các trang web được sử dụng để thu thập dữ liệu là IMDb, trang web này chứa rất nhiều trường thông tin, tuy nhiên việc khó khăn nhất trong quá trình thu thập dữ liệu đối với tôi chính là việc lấy các liên kết ảnh và liên kết trailer. Trên trang web này, cứ mỗi ngày các liên kết ảnh và trailer sẽ được làm mới một lần, do đó các liên kết cũ sẽ trở nên không hợp lệ và không thể truy cập được vào ngày hôm sau. Tôi đã đề xuất hai phương án để giải quyết vấn đề này: Tải toàn bộ ảnh và trailer về máy hoặc tìm cách để hệ thống tự động cập nhật dữ liệu.

Về phương án tải xuống toàn bộ hình ảnh và video trailer, tôi đã từ bỏ ngay khi quyết định số lượng phim cần thiết cho bản demo. Nếu có 5000 bộ phim thì sẽ có hơn 250000 hình ảnh và 5000 video demo, việc tải xuống sẽ khiến máy tính trở nên cồng kềnh và không hiệu quả. Vì vậy, tôi đã tìm hiểu và phát hiện ra một giải pháp để hệ thống có thể tự động cập nhật dữ liệu, đó là việc sử dụng GitHub Actions.

\textbf{Giới thiệu Github Action:}

GitHub Actions là một dịch vụ được GitHub cung cấp nhằm tự động hóa các quy trình phát triển phần mềm. Thông qua việc sử dụng các action, tôi có thể thiết lập các workflow tự động, bao gồm việc kiểm tra mã nguồn, triển khai ứng dụng, và tự động hóa các tác vụ như cập nhật dữ liệu trong hệ thống.

\textbf{Triển khai:}

Ở bước đầu tiên, tôi xây dựng một file crawlJson.py nhằm thu thập toàn bộ dữ liệu mới từ website IMDb và ghi vào file update.sql, một khi đã có file update.sql thì chỉ việc import vào môi trường local là có thể có dữ liệu mới ngay.

Nhằm cho phép hệ thống tự động cập nhật dữ liệu, quy trình thực hiện bao gồm các bước sau:

\begin{enumerate}
    \item Tạo PAT (Personal Access Token) Github

    Personal Access Token (PAT) trên GitHub đóng vai trò như một phương thức xác thực, cho phép các ứng dụng hoặc dịch vụ trung gian truy cập vào tài khoản mà không cần tiết lộ mật khẩu chính. Trong kiến trúc này, đối tượng sử dụng mã PAT chính là các bot vận hành tự động.

    \begin{figure}[H]
        \centering
        \includegraphics[width=1\linewidth]{Hinhve/gh_at_1.png}
        \caption{Tạo PAT Github}
        \label{fig:gh_at_1}
    \end{figure}
    
    \item Tạo Secrets Repository

    Việc thiết lập GitHub Secrets nhằm mục tiêu bảo mật và quản trị tập trung các dữ liệu nhạy cảm như mã token, mật khẩu, hay khóa API trong suốt vòng đời phát triển dự án. Cơ chế này đảm bảo những thông tin quan trọng không bị công khai trên mã nguồn, đồng thời chỉ cho phép các quy trình tự động hóa như GitHub Actions hoặc các ứng dụng được ủy quyền truy xuất thông qua API an toàn.

    \begin{figure}[H]
        \centering
        \includegraphics[width=1\linewidth]{Hinhve/gh_at_2.png}
        \caption{Tạo Secrets Repository}
        \label{fig:gh_at_2}
    \end{figure}
    
    \item Viết file update.yml

    Giai đoạn cuối cùng, tôi thiết lập tệp cấu hình update.yml nhằm kích hoạt cơ chế vận hành tự động định kỳ cho GitHub Actions.
\end{enumerate}

\textbf{Kết quả:}

Đúng 01:00 mỗi ngày, hệ thống sẽ tự động kích hoạt quy trình khai thác dữ liệu mới, kết xuất ra tệp update.sql và thực hiện cập nhật mã nguồn lên GitHub trong vòng một giờ đồng hồ.

\begin{figure}[H]
    \centering
    \includegraphics[width=1\linewidth]{Hinhve/gh_at_3.png}
    \caption{Hệ thống tự động crawl dữ liệu}
    \label{fig:gh_at_3}
\end{figure}

Ngay sau khi tệp update.sql được khởi tạo thành công, người dùng chỉ cần thực hiện thao tác kéo mã nguồn về máy chủ và khởi chạy tập lệnh Crawl/autoUpdate/update.py để cập nhật dữ liệu mới vào hệ thống tức thì.

\begin{figure}[H]
    \centering
    \includegraphics[width=1\linewidth]{Hinhve/gh_at_4.png}
    \caption{Cập nhật dữ liệu mới thành công}
    \label{fig:gh_at_4}
\end{figure}

\section{Lựa chọn thuật toán gợi ý phù hợp}

Việc lựa chọn thuật toán gợi ý phù hợp là một thách thức quan trọng trong quá trình phát triển ứng dụng gợi ý phim. Tôi đã tham khảo các bài báo và tài liệu trên mạng, thực hiện thử nghiệm và đánh giá các thuật toán khác nhau để tìm ra thuật toán phù hợp nhất với yêu cầu của dự án. Cụ thể chính là thuật toán Content-based.

\textbf{Giới thiệu thuật toán Content-based:}

Trong hệ thống content-based này, dựa trên nội dung của mỗi movie, tôi xây dựng một bộ hộ sơ cho mỗi movie. Hồ sơ này được biểu diễn dưới dạng toán học là một feature vector. Features của một bộ phim trong hệ thống gợi ý tôi xây dựng này là:

\begin{itemize}
    \item Movie name: Tên của bộ phim đó
    \item Genres: Thể loại của bộ phim đó
    \item Describe: Mô tả sơ lược về nội dung của bộ phim đó
    \item Cast: Diễn viên của bộ phim đó
    \item Director: Đạo diễn của bộ phim đó
\end{itemize}

\textbf{Phân tích, giải pháp:}

Content-based là thuật toán mà khi chọn một item, thì sẽ đưa ra các item tương tự có những đặc điểm cụ thể giống với item đã chọn. Vì thế trong bài toán này tôi đã chọn cách tính độ tương đồng giữa các bộ phim dựa trên các feature ở trên.

\[
\text{similarity}(A, B) = \cos(\theta) = \frac{A \cdot B}{\|A\| \|B\|}
\]

Trong đó:
\begin{itemize}
  \item \(A\) và \(B\) là hai vector đại diện cho hai phim hoặc hai người dùng.
  \item \(A \cdot B\) là tích vô hướng của hai vector \(A\) và \(B\).
  \item \(\|A\|\) và \(\|B\|\) lần lượt là độ lớn của vector \(A\) và vector \(B\).
\end{itemize}

\textbf{Kiểm tra tính chính xác:}

Sau khi tính toán được độ tương đồng dựa vào công thức similarity trên, tôi kiểm tra tính chính xác của thuật toán bằng cách: Tính toán độ tương đồng riêng lẻ của các đặc trưng như movie name, genres, cast, director, describe. Sau đó kiểm tra kết quả cuối cùng với các kết quả riêng biệt xem có chính xác không.

\begin{figure}[H]
    \centering
    \includegraphics[width=1\linewidth]{Hinhve/rcm_ct_kq_1.png}
    \caption{Kiểm tra kết quả 1}
    \label{fig:rcm_ct_kq_1}
\end{figure}

Từ hình ảnh \ref{fig:rcm_ct_kq_1}, có thể nhận thấy rằng bộ phim được hệ thống đề xuất có mức độ tương đồng cao nhất với bộ phim Guardians of the Galaxy chính là bộ phim Guardians of the Galaxy Vol. 3. Hai bộ phim này có sự tương đồng về tên phim, mô tả, diễn viên và đạo diễn.

\begin{figure}[H]
    \centering
    \includegraphics[width=1\linewidth]{Hinhve/rcm_ct_kq_2.png}
    \caption{Kiểm tra kết quả 2}
    \label{fig:rcm_ct_kq_2}
\end{figure}

Từ hình ảnh \ref{fig:rcm_ct_kq_2}, có thể quan sát thấy bộ phim được hệ thống đề xuất có mức độ tương đồng cao nhất với bộ phim The Hunger Games: Catching Fire là bộ phim The Hunger Games. Hai bộ phim này thể hiện sự tương đồng đáng kể về tên phim, mô tả, diễn viên và đạo diễn.


\begin{figure}[H]
    \centering
    \includegraphics[width=1\linewidth]{Hinhve/rcm_ct_kq_3.png}
    \caption{Kiểm tra kết quả 3}
    \label{fig:rcm_ct_kq_3}
\end{figure}

Dựa trên kết quả hiển thị tại hình \ref{fig:rcm_ct_kq_3}, có thể thấy The Meg là tác phẩm được hệ thống đánh giá có mức độ tương quan cao nhất đối với bộ phim Meg 2: The Trench. Sự tương đồng này được thể hiện rõ nét qua các yếu tố như tiêu đề, cốt truyện mô tả, cùng sự trùng khớp về đội ngũ diễn viên và đạo diễn.

Dựa trên các kết quả đã được trình bày ở trên kết hợp với việc kiểm tra bằng phương pháp trực quan, có thể kết luận rằng hệ thống gợi ý phim dựa trên độ tương đồng đã được thực hiện với mức độ chính xác cao.

\end{document}