\documentclass[../DoAn.tex]{subfiles}
\begin{document}
\section{Công nghệ sử dụng phía frontend}
Chương này được dành để giới thiệu về các công nghệ và các nền tảng kỹ thuật chính đã được sử dụng trong việc phát triển hệ thống gợi ý phim có tích hợp genAI. Những công nghệ này đóng một vai trò quan trọng và không thể thiếu trong toàn bộ quá trình xây dựng, triển khai và cải thiện hiệu suất của hệ thống, nhằm đảm bảo rằng giải pháp được đề xuất có tính khả thi và đạt được hiệu quả như mong đợi.

\label{section:4.1}
\subsection{React}
React là một thư viện JavaScript mã nguồn mở được phát triển bởi Meta, được thiết kế để xây dựng giao diện người dùng dựa trên mô hình component. React vận dụng cơ chế Virtual DOM nhằm tối ưu hóa quá trình cập nhật giao diện, tạo điều kiện để phát triển các ứng dụng web có tính động và tương tác cao.

Lý do tôi lựa chọn React là do nó cho phép phân tách giao diện thành các thành phần độc lập, tạo điều kiện thuận lợi cho việc quản lý, bảo trì và tái sử dụng mã nguồn. Bên cạnh đó, cơ chế ràng buộc dữ liệu một chiều (one-way data binding) cùng với hệ sinh thái đa dạng của React góp phần làm cho quá trình phát triển các tính năng trở nên linh hoạt và đạt hiệu quả cao hơn. Nhờ vào cơ chế Virtual DOM, React đảm bảo ứng dụng vận hành mượt mà ngay cả trong trường hợp phải xử lý khối lượng dữ liệu phim ảnh lớn và các tương tác phức tạp từ phía người dùng.

Trong dự án này, React được áp dụng để phát triển toàn bộ phần giao diện frontend, bao gồm các chức năng như tìm kiếm phim, lọc phim, hiển thị danh sách phim, trang thông tin chi tiết phim, và các chức năng tương tác khác. Các component React được tổ chức theo mô hình module, góp phần làm cho mã nguồn trở nên dễ đọc và dễ bảo trì hơn.

\subsection{Tailwind CSS}
Tailwind CSS là một framework CSS được xây dựng theo triết lý utility-first, cung cấp các lớp tiện ích cấp thấp để tạo giao diện trực tiếp trong mã nguồn mà không cần phải rời khỏi các tệp HTML hoặc JSX. Khác biệt so với các framework CSS truyền thống, Tailwind CSS không áp đặt các thành phần giao diện có sẵn, tạo điều kiện để người phát triển có thể kiểm soát thiết kế một cách tự do và linh hoạt.

Lý do tôi lựa chọn Tailwind CSS là do nó có khả năng tối ưu hóa kích thước tệp CSS cuối cùng thông qua cơ chế loại bỏ các đoạn mã không được sử dụng, đồng thời hỗ trợ việc thiết kế giao diện đáp ứng một cách trực quan và nhanh chóng. Framework này cho phép tôi có toàn quyền kiểm soát thiết kế và tùy chỉnh giao diện hệ thống mà không bị giới hạn bởi các thành phần có sẵn.

Trong dự án này, Tailwind CSS được áp dụng để xây dựng toàn bộ phần styling cho giao diện, bao gồm các component như nút bấm, form nhập liệu, card hiển thị phim, và bố cục responsive. Tailwind CSS đảm bảo giao diện có tính hiện đại, tinh tế và tương thích tốt trên đa dạng các loại thiết bị, từ máy tính để bàn cho đến các thiết bị di động.

\section{Công nghệ sử dụng phía backend}
\label{section:4.2}
\subsection{Django}
Django là một framework phát triển web được xây dựng trên nền tảng ngôn ngữ lập trình Python, nổi tiếng với khả năng tạo ra các ứng dụng web trong thời gian ngắn và đạt được hiệu suất làm việc tốt. Framework này được thiết kế với kiến trúc MVT (Model-View-Template) có tính nhất quán và dễ hiểu, giúp cho việc sắp xếp và quản lý mã nguồn trở nên có hệ thống và hợp lý hơn. Django có khả năng kết nối với nhiều loại hệ quản trị cơ sở dữ liệu khác nhau và được tích hợp sẵn các công cụ mạnh mẽ như ORM cùng với hệ thống quản trị admin được tự động hóa.

Việc tôi quyết định sử dụng Django xuất phát từ khả năng của nó trong việc tương tác với cơ sở dữ liệu thông qua ORM mà không đòi hỏi phải viết các câu lệnh SQL một cách thủ công, nhờ đó có thể rút ngắn thời gian phát triển và hạn chế các sai sót trong quá trình lập trình. Ngoài ra, Django còn được tích hợp sẵn các cơ chế bảo mật nhằm ngăn chặn các hình thức tấn công phổ biến như CSRF, XSS và SQL Injection, từ đó nâng cao mức độ an toàn của hệ thống. Framework này còn có một cộng đồng người dùng và nhà phát triển đông đảo cùng với hệ thống tài liệu phong phú, tạo thuận lợi cho việc phát triển dự án và xử lý các vấn đề kỹ thuật khi phát sinh.

Trong dự án này, Django được áp dụng để phát triển toàn bộ phần backend, bao gồm các API endpoints để xử lý các yêu cầu từ frontend, quản lý dữ liệu về phim và người dùng thông qua ORM, xử lý logic nghiệp vụ cho các chức năng như tìm kiếm, gợi ý phim, và quản lý đánh giá. Django REST Framework được sử dụng kết hợp để xây dựng RESTful API, tạo điều kiện cho việc giao tiếp giữa frontend React và backend Django.

\subsection{MySQL}
MySQL là một hệ thống quản lý cơ sở dữ liệu quan hệ có mã nguồn mở, được sử dụng phổ biến trong nhiều ứng dụng web do đặc tính ổn định và khả năng xử lý hiệu quả. MySQL được tích hợp các chức năng mạnh mẽ bao gồm quản lý các dữ liệu có độ phức tạp cao, thực thi các truy vấn với hiệu suất cao, đảm bảo mức độ bảo mật tốt, và có khả năng mở rộng dễ dàng. Hệ thống này hoạt động dựa trên ngôn ngữ SQL tiêu chuẩn và có khả năng hỗ trợ nhiều loại dữ liệu khác nhau.

Lý do tôi lựa chọn MySQL là do nó phù hợp với các yêu cầu về lưu trữ và truy xuất dữ liệu phim ảnh của ứng dụng, đảm bảo dữ liệu được tổ chức một cách có cấu trúc và dễ dàng quản lý. MySQL có khả năng sao lưu và phục hồi dữ liệu hiệu quả, góp phần bảo vệ các dữ liệu quan trọng của ứng dụng. Bên cạnh đó, MySQL có khả năng tích hợp tốt với Django thông qua ORM, tạo điều kiện thuận lợi cho quá trình phát triển.

Trong dự án này, MySQL được áp dụng để lưu trữ toàn bộ dữ liệu của hệ thống, bao gồm các thông tin về phim (tên phim, mô tả, thể loại, năm sản xuất), thông tin về đạo diễn, diễn viên, giải thưởng, dữ liệu người dùng, đánh giá phim, và lịch sử hoạt động của người dùng. Cơ sở dữ liệu này đảm bảo dữ liệu được lưu trữ một cách an toàn và có thể truy xuất hiệu quả cho các chức năng như tìm kiếm, gợi ý phim, và phân tích hành vi người dùng.

\section{Công nghệ sử dụng trong việc lưu trữ code}
\label{section:4.3}
\subsection{Git}
Git là một công cụ quản lý phiên bản hoạt động theo mô hình phân tán, có khả năng ghi nhận và theo dõi mọi thay đổi trong mã nguồn trong toàn bộ vòng đời phát triển dự án. Git lưu trữ toàn bộ lịch sử các thay đổi dưới hình thức các snapshot, cho phép người dùng có thể khôi phục lại bất kỳ phiên bản nào đã từng tồn tại trước đó. Công cụ này được thiết kế dựa trên nguyên lý phân tán, trong đó mỗi bản sao của repository đều lưu trữ đầy đủ toàn bộ lịch sử phát triển của dự án.

Quyết định của tôi trong việc chọn Git dựa trên khả năng quản lý mã nguồn hiệu quả của nó, giúp cho việc khôi phục các phiên bản cũ trở nên đơn giản khi xảy ra lỗi hoặc khi cần đối chiếu các thay đổi. Git được tích hợp các chức năng quan trọng như tạo nhánh, gộp mã và ghi chép lịch sử, đóng góp vào việc điều phối quá trình phát triển và giải quyết các mâu thuẫn trong mã nguồn một cách có hiệu quả. Hơn nữa, Git tạo điều kiện cho việc hợp tác nhóm được diễn ra suôn sẻ, cho phép mỗi thành viên có thể làm việc một cách độc lập trên các nhánh riêng của mình.

Trong dự án này, Git được áp dụng để quản lý toàn bộ mã nguồn của dự án, bao gồm mã frontend React, backend Django, các script crawl dữ liệu, và các tệp cấu hình. Tôi sử dụng các nhánh khác nhau để phát triển các tính năng riêng biệt, sau đó hợp nhất vào nhánh chính sau khi hoàn thành và kiểm thử. Git góp phần theo dõi mọi thay đổi trong quá trình phát triển và đảm bảo mã nguồn luôn được sao lưu một cách an toàn.

\subsection{GitHub}
GitHub là một dịch vụ lưu trữ mã nguồn được phát triển trên nền tảng Git, mang đến một môi trường trực tuyến để trao đổi, lưu giữ và điều hành mã nguồn. GitHub cung cấp một giao diện web dễ sử dụng để quản lý các kho chứa mã, giám sát các vấn đề kỹ thuật, xử lý các yêu cầu tính năng mới và thực hiện việc xem xét lại mã nguồn. Nền tảng này còn được bổ sung nhiều công cụ bổ trợ như hệ thống quản lý dự án, wiki, và các tính năng tích hợp liên tục CI/CD.

Lý do tôi lựa chọn GitHub là do nó tạo điều kiện để dễ dàng chia sẻ mã nguồn và phối hợp phát triển sản phẩm một cách hiệu quả, đặc biệt trong trường hợp làm việc với nhóm hoặc cần sao lưu mã nguồn trên đám mây. GitHub cung cấp các tính năng review mã và quản lý dự án, góp phần làm cho quy trình phát triển trở nên chuyên nghiệp và linh hoạt hơn. Bên cạnh đó, GitHub hỗ trợ tích hợp với đa dạng các dịch vụ và công cụ khác, tạo điều kiện để tự động hóa các quy trình phát triển và triển khai sản phẩm thông qua GitHub Actions.

Trong dự án này, GitHub được áp dụng để lưu trữ toàn bộ mã nguồn của dự án trên đám mây, đảm bảo mã nguồn luôn được sao lưu và có thể truy cập từ bất kỳ đâu. Tôi sử dụng GitHub Issues để theo dõi các công việc cần thực hiện và các lỗi cần được khắc phục. Ngoài ra, GitHub Actions được áp dụng để tự động hóa quy trình cập nhật dữ liệu phim hàng ngày, thực thi các script crawl dữ liệu từ IMDb và BoxOfficeMojo.

\section{Công nghệ sử dụng trong việc crawl và update dữ liệu}
Trong quá trình phát triển ứng dụng gợi ý phim bằng genAI, cần thu thập và duy trì cập nhật dữ liệu từ các nguồn có độ tin cậy cao để đảm bảo độ chính xác và tính cập nhật của các thông tin về phim ảnh. Nhằm thực hiện nhiệm vụ này, tôi đã sử dụng các công cụ có khả năng mạnh mẽ như Selenium và Requests, đồng thời kết hợp với việc tự động hóa toàn bộ quy trình cập nhật dữ liệu bằng cách sử dụng GitHub Actions.

\label{section:4.4}
\subsection{Selenium và Requests}
Selenium là một công cụ tự động hóa trình duyệt web có khả năng mạnh mẽ, cho phép điều khiển trình duyệt và thực hiện các tương tác với các trang web có nội dung động giống như cách một người dùng thực tế sẽ thao tác. Selenium có thể thực hiện nhiều loại hành động khác nhau bao gồm click chuột, nhập dữ liệu, cuộn trang, và chờ đợi các thành phần động được tải hoàn tất. Với thiết kế tinh gọn nhưng hiệu quả, Requests là thư viện chuyên dụng trong Python giúp tối ưu hóa việc truyền tải các yêu cầu HTTP và khai thác dữ liệu từ các nền tảng trực tuyến. Requests được tích hợp nhiều chức năng hữu ích như điều hướng qua các trang web, gửi dữ liệu dưới định dạng form và quản lý cookies.

Việc tôi quyết định sử dụng Selenium và Requests bắt nguồn từ khả năng của chúng trong việc tự động thu thập dữ liệu từ các trang web có độ tin cậy cao như IMDb và BoxOfficeMojo một cách hiệu quả. Selenium được sử dụng để xử lý các trang web có nội dung động đòi hỏi người dùng phải thực hiện các thao tác như cuộn trang hoặc nhấn nút để tải thêm nội dung, còn Requests được áp dụng cho các trang web tĩnh để lấy dữ liệu với tốc độ cao hơn. Việc kết hợp sử dụng hai công cụ này đã giúp tôi có thể thu thập được một lượng dữ liệu phong phú và đáng tin cậy về phim ảnh để phục vụ cho hệ thống.

Trong dự án này, Selenium được áp dụng để crawl dữ liệu từ IMDb, bao gồm các thông tin về phim (tên phim, mô tả, thể loại, năm sản xuất, đánh giá), thông tin về đạo diễn, diễn viên, giải thưởng, và hình ảnh. Requests được sử dụng để lấy dữ liệu tĩnh từ BoxOfficeMojo về doanh thu phòng vé của các bộ phim. Các script crawl này được thực thi tự động hàng ngày thông qua GitHub Actions để đảm bảo dữ liệu luôn được cập nhật mới nhất.

\subsection{Công nghệ sử dụng để cập nhật dữ liệu}
Việc liên tục đồng bộ và làm mới cơ sở dữ liệu điện ảnh là yếu tố then chốt giúp thuật toán đề xuất duy trì tính thời sự cũng như độ chính xác cao cho các nội dung phản hồi tới người dùng. Nhằm thực hiện tự động hóa quy trình này, tôi đã sử dụng GitHub Actions. GitHub Actions là một công cụ CI/CD được tích hợp sẵn trong nền tảng GitHub, có khả năng cho phép người dùng tạo ra các workflow tự động sẽ được kích hoạt khi xảy ra các sự kiện nhất định.

Bằng cách sử dụng GitHub Actions, tôi đã tạo ra các workflow có khả năng tự động cập nhật dữ liệu phim từ IMDb theo chu kỳ hàng ngày. Những quy trình tự động này vận hành các tập lệnh Selenium và Requests nhằm khai thác thông tin mới, từ đó đồng bộ hóa và làm mới dữ liệu trực tiếp vào hệ thống lưu trữ. Quá trình tự động hóa này đảm bảo rằng dữ liệu luôn được cập nhật đều đặn theo đúng lịch trình đã định mà không cần có sự can thiệp từ con người.

Cụ thể hơn, mỗi ngày GitHub Actions sẽ tự động khởi chạy các script thu thập dữ liệu để kiểm tra và áp dụng các thay đổi mới nhất từ IMDb. Hệ thống thực hiện việc bổ sung và làm mới các thuộc tính như nguồn thu, đoạn giới thiệu, tư liệu hình ảnh cùng những dữ liệu phụ trợ liên quan thông qua quy trình GitHub Actions được thiết lập theo các giai đoạn cụ thể dưới đây:

Thông qua thư viện Requests, hệ thống thực hiện truyền tải các lệnh gọi HTTP nhằm trích xuất các nguồn tài nguyên tĩnh từ những nền tảng web mục tiêu.

Thực hiện việc đọc, xử lý và làm sạch các dữ liệu đã được thu thập, đảm bảo rằng dữ liệu có đúng định dạng và cấu trúc yêu cầu.

Dữ liệu sau khi xử lý sẽ được đồng bộ trực tiếp vào hệ thống MySQL để đảm bảo tính cập nhật liên tục.

Việc tích hợp GitHub Actions không chỉ tối ưu hóa khả năng tự động hóa quy trình làm mới dữ liệu mà còn duy trì sự nhất quán và độ tin cậy cho kho thông tin điện ảnh. Cơ chế này hỗ trợ hệ thống đề xuất luôn phản hồi những kết quả mới nhất, từ đó đáp ứng trọn vẹn nhu cầu truy vấn của người dùng.

Phần này đã tổng hợp các giải pháp công nghệ cốt lõi được ứng dụng trong việc xây dựng hệ thống gợi ý phim. Sự kết hợp giữa Django, MySQL ở phía server cùng React và Tailwind CSS cho giao diện, đi kèm với quy trình quản lý mã nguồn qua Git/GitHub và phương thức khai thác dữ liệu từ IMDb, BoxOfficeMojo đã tạo nên một nền tảng vận hành ổn định, hiệu quả. Những lựa chọn kỹ thuật này không chỉ giải quyết tốt các yêu cầu hiện tại mà còn tạo tiền đề vững chắc cho việc nâng cấp và mở rộng quy mô trong tương lai.

\end{document}