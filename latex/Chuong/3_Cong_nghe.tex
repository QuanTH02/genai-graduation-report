\documentclass[../DoAn.tex]{subfiles}
\begin{document}
\section{Công nghệ sử dụng phía frontend}
Chương này trình bày các công nghệ và nền tảng kỹ thuật đã được áp dụng trong quá trình phát triển ứng dụng gợi ý phim tích hợp genAI. Các công nghệ này có vai trò then chốt trong việc xây dựng, triển khai và tối ưu hóa hệ thống, góp phần đảm bảo tính khả thi và hiệu quả của giải pháp được đề xuất.

\label{section:4.1}
\subsection{React}
React là một thư viện JavaScript mã nguồn mở được phát triển bởi Meta, được thiết kế để xây dựng giao diện người dùng dựa trên mô hình component. React vận dụng cơ chế Virtual DOM nhằm tối ưu hóa quá trình cập nhật giao diện, tạo điều kiện để phát triển các ứng dụng web có tính động và tương tác cao.

Lý do tôi lựa chọn React là do nó cho phép phân tách giao diện thành các thành phần độc lập, tạo điều kiện thuận lợi cho việc quản lý, bảo trì và tái sử dụng mã nguồn. Bên cạnh đó, cơ chế ràng buộc dữ liệu một chiều (one-way data binding) cùng với hệ sinh thái đa dạng của React góp phần làm cho quá trình phát triển các tính năng trở nên linh hoạt và đạt hiệu quả cao hơn. Nhờ vào cơ chế Virtual DOM, React đảm bảo ứng dụng vận hành mượt mà ngay cả trong trường hợp phải xử lý khối lượng dữ liệu phim ảnh lớn và các tương tác phức tạp từ phía người dùng.

Trong dự án này, React được áp dụng để phát triển toàn bộ phần giao diện frontend, bao gồm các chức năng như tìm kiếm phim, lọc phim, hiển thị danh sách phim, trang thông tin chi tiết phim, và các chức năng tương tác khác. Các component React được tổ chức theo mô hình module, góp phần làm cho mã nguồn trở nên dễ đọc và dễ bảo trì hơn.

\subsection{Tailwind CSS}
Tailwind CSS là một framework CSS được xây dựng theo triết lý utility-first, cung cấp các lớp tiện ích cấp thấp để tạo giao diện trực tiếp trong mã nguồn mà không cần phải rời khỏi các tệp HTML hoặc JSX. Khác biệt so với các framework CSS truyền thống, Tailwind CSS không áp đặt các thành phần giao diện có sẵn, tạo điều kiện để người phát triển có thể kiểm soát thiết kế một cách tự do và linh hoạt.

Lý do tôi lựa chọn Tailwind CSS là do nó có khả năng tối ưu hóa kích thước tệp CSS cuối cùng thông qua cơ chế loại bỏ các đoạn mã không được sử dụng, đồng thời hỗ trợ việc thiết kế giao diện đáp ứng một cách trực quan và nhanh chóng. Framework này cho phép tôi có toàn quyền kiểm soát thiết kế và tùy chỉnh giao diện hệ thống mà không bị giới hạn bởi các thành phần có sẵn.

Trong dự án này, Tailwind CSS được áp dụng để xây dựng toàn bộ phần styling cho giao diện, bao gồm các component như nút bấm, form nhập liệu, card hiển thị phim, và bố cục responsive. Tailwind CSS đảm bảo giao diện có tính hiện đại, tinh tế và tương thích tốt trên đa dạng các loại thiết bị, từ máy tính để bàn cho đến các thiết bị di động.

\section{Công nghệ sử dụng phía backend}
\label{section:4.2}
\subsection{Django}
Django là một framework phát triển ứng dụng web được xây dựng bằng ngôn ngữ Python, được biết đến với khả năng hỗ trợ xây dựng các ứng dụng web một cách nhanh chóng và đạt hiệu quả cao. Django cung cấp một kiến trúc MVT (Model-View-Template) rõ ràng và nhất quán, tạo điều kiện để tổ chức mã nguồn một cách khoa học và có logic. Framework này hỗ trợ tích hợp với đa dạng các hệ quản trị cơ sở dữ liệu và được trang bị nhiều tính năng mạnh mẽ như ORM và hệ thống quản trị admin tự động.

Lý do tôi lựa chọn Django là do nó tạo điều kiện thuận lợi để thao tác với cơ sở dữ liệu thông qua ORM mà không cần phải viết các câu lệnh SQL thủ công, từ đó giảm thiểu thời gian phát triển và các lỗi lập trình có thể xảy ra. Bên cạnh đó, Django còn được tích hợp sẵn nhiều tính năng bảo mật như bảo vệ chống lại các cuộc tấn công CSRF, XSS và SQL Injection, góp phần đảm bảo tính an toàn cho hệ thống. Framework này cũng sở hữu một cộng đồng phát triển lớn và tài liệu đầy đủ, tạo điều kiện thuận lợi cho việc phát triển và giải quyết các vấn đề phát sinh.

Trong dự án này, Django được áp dụng để phát triển toàn bộ phần backend, bao gồm các API endpoints để xử lý các yêu cầu từ frontend, quản lý dữ liệu về phim và người dùng thông qua ORM, xử lý logic nghiệp vụ cho các chức năng như tìm kiếm, gợi ý phim, và quản lý đánh giá. Django REST Framework được sử dụng kết hợp để xây dựng RESTful API, tạo điều kiện cho việc giao tiếp giữa frontend React và backend Django.

\subsection{MySQL}
MySQL là một hệ quản trị cơ sở dữ liệu quan hệ mã nguồn mở, được áp dụng rộng rãi trong các ứng dụng web nhờ vào tính ổn định và hiệu suất hoạt động cao. MySQL được trang bị các tính năng mạnh mẽ như khả năng quản lý dữ liệu phức tạp, thực hiện các truy vấn với tốc độ cao, mức độ bảo mật tốt, và khả năng mở rộng linh hoạt. Hệ quản trị này sử dụng ngôn ngữ SQL chuẩn và hỗ trợ đa dạng các kiểu dữ liệu khác nhau.

Lý do tôi lựa chọn MySQL là do nó phù hợp với các yêu cầu về lưu trữ và truy xuất dữ liệu phim ảnh của ứng dụng, đảm bảo dữ liệu được tổ chức một cách có cấu trúc và dễ dàng quản lý. MySQL có khả năng sao lưu và phục hồi dữ liệu hiệu quả, góp phần bảo vệ các dữ liệu quan trọng của ứng dụng. Bên cạnh đó, MySQL có khả năng tích hợp tốt với Django thông qua ORM, tạo điều kiện thuận lợi cho quá trình phát triển.

Trong dự án này, MySQL được áp dụng để lưu trữ toàn bộ dữ liệu của hệ thống, bao gồm các thông tin về phim (tên phim, mô tả, thể loại, năm sản xuất), thông tin về đạo diễn, diễn viên, giải thưởng, dữ liệu người dùng, đánh giá phim, và lịch sử hoạt động của người dùng. Cơ sở dữ liệu này đảm bảo dữ liệu được lưu trữ một cách an toàn và có thể truy xuất hiệu quả cho các chức năng như tìm kiếm, gợi ý phim, và phân tích hành vi người dùng.

\section{Công nghệ sử dụng trong việc lưu trữ code}
\label{section:4.3}
\subsection{Git}
Git là một hệ thống quản lý phiên bản phân tán, cho phép theo dõi các thay đổi của mã nguồn trong suốt quá trình phát triển. Git lưu trữ lịch sử thay đổi của mã nguồn dưới dạng các snapshot, tạo điều kiện để quay lại bất kỳ phiên bản nào trước đó. Hệ thống này hoạt động dựa trên mô hình phân tán, trong đó mỗi bản sao repository đều chứa đầy đủ lịch sử của dự án.

Lý do tôi lựa chọn Git là do nó hỗ trợ quản lý mã nguồn một cách hiệu quả, tạo điều kiện để dễ dàng quay lại các phiên bản trước đó khi gặp lỗi hoặc cần so sánh các thay đổi. Git được trang bị các tính năng mạnh mẽ như nhánh, hợp nhất và theo dõi lịch sử, góp phần quản lý quá trình phát triển và xử lý các xung đột mã nguồn một cách hiệu quả. Bên cạnh đó, Git cho phép phối hợp làm việc nhóm một cách mượt mà, mỗi thành viên có thể làm việc độc lập trên các nhánh riêng biệt.

Trong dự án này, Git được áp dụng để quản lý toàn bộ mã nguồn của dự án, bao gồm mã frontend React, backend Django, các script crawl dữ liệu, và các tệp cấu hình. Tôi sử dụng các nhánh khác nhau để phát triển các tính năng riêng biệt, sau đó hợp nhất vào nhánh chính sau khi hoàn thành và kiểm thử. Git góp phần theo dõi mọi thay đổi trong quá trình phát triển và đảm bảo mã nguồn luôn được sao lưu một cách an toàn.

\subsection{GitHub}
GitHub là một dịch vụ lưu trữ mã nguồn được xây dựng dựa trên Git, cung cấp một nền tảng trực tuyến để chia sẻ, lưu trữ và quản lý mã nguồn. GitHub cung cấp giao diện web thân thiện với người dùng để quản lý các kho lưu trữ, theo dõi các vấn đề, yêu cầu tính năng và xem lại mã nguồn. Nền tảng này cũng được trang bị nhiều công cụ hỗ trợ như quản lý dự án, wiki, và tích hợp liên tục CI/CD.

Lý do tôi lựa chọn GitHub là do nó tạo điều kiện để dễ dàng chia sẻ mã nguồn và phối hợp phát triển sản phẩm một cách hiệu quả, đặc biệt trong trường hợp làm việc với nhóm hoặc cần sao lưu mã nguồn trên đám mây. GitHub cung cấp các tính năng review mã và quản lý dự án, góp phần làm cho quy trình phát triển trở nên chuyên nghiệp và linh hoạt hơn. Bên cạnh đó, GitHub hỗ trợ tích hợp với đa dạng các dịch vụ và công cụ khác, tạo điều kiện để tự động hóa các quy trình phát triển và triển khai sản phẩm thông qua GitHub Actions.

Trong dự án này, GitHub được áp dụng để lưu trữ toàn bộ mã nguồn của dự án trên đám mây, đảm bảo mã nguồn luôn được sao lưu và có thể truy cập từ bất kỳ đâu. Tôi sử dụng GitHub Issues để theo dõi các công việc cần thực hiện và các lỗi cần được khắc phục. Ngoài ra, GitHub Actions được áp dụng để tự động hóa quy trình cập nhật dữ liệu phim hàng ngày, thực thi các script crawl dữ liệu từ IMDb và BoxOfficeMojo.

\section{Công nghệ sử dụng trong việc crawl dữ liệu}
Trong quá trình xây dựng ứng dụng gợi ý phim, việc thu thập và cập nhật dữ liệu từ các nguồn đáng tin cậy là một yếu tố quan trọng nhằm đảm bảo tính chính xác và tính cập nhật của thông tin phim ảnh. Để thực hiện công việc này, tôi đã áp dụng các công cụ mạnh mẽ như Selenium và Requests, kết hợp với việc tự động hóa quy trình cập nhật dữ liệu thông qua GitHub Actions.

\label{section:4.4}
\subsection{Selenium và Requests}
Selenium là một công cụ tự động hóa web mạnh mẽ, cho phép điều khiển trình duyệt web và tương tác với các trang web động tương tự như cách người dùng thực tế thao tác. Selenium có khả năng thực hiện các hành động như click, nhập liệu, cuộn trang, và chờ đợi các phần tử động được tải. Requests là một thư viện HTTP đơn giản nhưng mạnh mẽ trong Python, hỗ trợ gửi các yêu cầu HTTP và truy xuất dữ liệu từ các trang web. Requests được trang bị nhiều tính năng như duyệt qua các trang web, gửi dữ liệu dưới dạng biểu mẫu và xử lý cookies.

Lý do tôi lựa chọn Selenium và Requests là do chúng tạo điều kiện để thu thập dữ liệu từ các trang web đáng tin cậy như IMDb và BoxOfficeMojo một cách tự động và đạt hiệu quả cao. Selenium được áp dụng để xử lý các trang web động yêu cầu người dùng phải cuộn trang hoặc nhấn nút để tải thêm dữ liệu, trong khi Requests được sử dụng cho các trang web tĩnh để truy xuất dữ liệu với tốc độ nhanh hơn. Sự kết hợp của hai công cụ này góp phần giúp tôi có thể thu thập được dữ liệu đa dạng và chính xác về phim ảnh cho ứng dụng.

Trong dự án này, Selenium được áp dụng để crawl dữ liệu từ IMDb, bao gồm các thông tin về phim (tên phim, mô tả, thể loại, năm sản xuất, đánh giá), thông tin về đạo diễn, diễn viên, giải thưởng, và hình ảnh. Requests được sử dụng để lấy dữ liệu tĩnh từ BoxOfficeMojo về doanh thu phòng vé của các bộ phim. Các script crawl này được thực thi tự động hàng ngày thông qua GitHub Actions để đảm bảo dữ liệu luôn được cập nhật mới nhất.

\subsection{Công nghệ sử dụng để update dữ liệu}
Việc duy trì và cập nhật dữ liệu phim ảnh một cách liên tục là một yêu cầu cần thiết để ứng dụng gợi ý phim luôn cung cấp thông tin mới nhất và chính xác. Để tự động hóa quy trình này, tôi đã áp dụng GitHub Actions. GitHub Actions là một công cụ CI/CD được tích hợp sẵn trong GitHub, cho phép thiết lập các workflow tự động thực thi khi có các sự kiện cụ thể.

Thông qua việc sử dụng GitHub Actions, tôi đã thiết lập các workflow để tự động cập nhật dữ liệu phim từ IMDb hàng ngày. Các workflow này bao gồm việc thực thi các script Selenium và Requests để thu thập dữ liệu mới, sau đó cập nhật vào cơ sở dữ liệu của ứng dụng. Việc tự động hóa này góp phần đảm bảo dữ liệu luôn được cập nhật một cách liên tục theo lịch trình mà không cần sự can thiệp thủ công.

Cụ thể, mỗi ngày GitHub Actions sẽ kích hoạt các script crawl dữ liệu tự động để kiểm tra và cập nhật những thay đổi mới nhất trên IMDb. Điều này bao gồm các trường thông tin như doanh thu, trailer và hình ảnh cùng các thông tin liên quan khác. Workflow GitHub Actions được tôi cài đặt với các bước như sau:

Sử dụng thư viện Requests để gửi các yêu cầu HTTP và lấy dữ liệu tĩnh từ các trang web liên quan.

Đọc, xử lý và làm sạch dữ liệu đã thu thập được, đảm bảo dữ liệu đúng định dạng và cấu trúc.

Cập nhật vào cơ sở dữ liệu MySQL với các thông tin mới nhất.

Việc áp dụng GitHub Actions không những góp phần tự động hóa quy trình cập nhật dữ liệu mà còn đảm bảo tính nhất quán và chính xác của thông tin phim ảnh trong ứng dụng. Điều này góp phần giúp ứng dụng gợi ý phim luôn cung cấp các gợi ý và thông tin mới nhất, đáp ứng nhu cầu của người dùng một cách tối ưu.

Chương này đã trình bày các công nghệ và nền tảng kỹ thuật đã được áp dụng trong quá trình phát triển ứng dụng gợi ý phim. Việc lựa chọn Django và MySQL cho phần backend, React và Tailwind CSS cho phần frontend, Git và GitHub cho việc lưu trữ mã nguồn, cùng với các kỹ thuật crawl dữ liệu từ IMDb và BoxOfficeMojo đã góp phần giúp tôi xây dựng một ứng dụng gợi ý phim có hiệu quả, ổn định và thân thiện với người dùng. Các công nghệ này không những đáp ứng các yêu cầu của ứng dụng mà còn tạo điều kiện thuận lợi cho việc mở rộng và nâng cấp trong tương lai.

\end{document}